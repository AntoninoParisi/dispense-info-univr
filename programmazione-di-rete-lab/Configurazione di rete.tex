\documentclass[a4paper, titlepage]{article}

\usepackage[italian]{babel}
\usepackage[sfdefault]{ClearSans}
\usepackage[T1]{fontenc}
\usepackage[utf8]{inputenc}
\usepackage{lipsum}
\usepackage[scaled=1.04]{couriers}
\usepackage{multicol}
\usepackage{fullpage}
\usepackage{listings}
\usepackage[top=1in, bottom=1in, left=0.5in, right=0.5in]{geometry}

\setlength{\columnsep}{1.5cm}
\setlength{\columnseprule}{0.4pt}
\lstset{literate={á}{{\'a}}1 {ã}{{\~a}}1 {è}{{\'e}}1,
basicstyle=\ttfamily}

\pagenumbering{gobble}% Remove page numbers (and reset to 1)
\clearpage
\pagestyle{empty}
\begin{document}
\begin{center}
\section*{\Huge Configurazione di Rete}
\end{center}
\begin{multicols}{2}
\subsection*{Configurazione switch}
	\begin{lstlisting}
enable
#configure terminal
#hostname SW1
#vlan 10
#name studenti
exit

show vlan	
	\end{lstlisting}
\subsection*{Assegnamento VLAN alle porte switch}
	\begin{lstlisting}
#configure terminal
#interface F0/1
#switchport access vlan 10
exit
	\end{lstlisting}
\subsection*{Porte Trunk o up-link}
	\begin{lstlisting}
#configure terminal
#interface F0/3
#switchport mode trunk
exit	
	\end{lstlisting}
\subsection*{Configurazione Router}
	\begin{lstlisting}
enable
#configure terminal
#interface FastEthernet0/0.1
#encapsulation dot1q 10
#ipaddress 192.168.0.1 255.255.255.0
//per le altre interfacce:
#interface FastEthernet0/0.2
#encapsulation dot1q 20
#ipaddress 192.168.1.1 255.255.255.0
...
#interface FastEthernet 0/0
#no shutdown
exit

show controllers S0/0
// se è un cavo dalla parte DCE
#interface S0/0
#clock rate 64000
exit

	\end{lstlisting}
\subsection*{DHCP}
	\begin{lstlisting}
#ip dhcp pool studenti
#network 192.168.0.0 255.255.255.0
#default-router 192.168.0.1
#lease 7
exit

ipconfig/all	
	\end{lstlisting}	
\subsection*{RIP}
	\begin{lstlisting}
#configure terminal
#router rip
#network 10.0.0.0
#network 10.0.1.0
#network 192.168.0.0
#network 192.168.1.0
exit
	\end{lstlisting}
\subsection*{NAT}
	\begin{lstlisting}
//interfaccia F0/0 rivolta all'interno
//interfaccia F0/1 è pubblica
#configure terminal
#interface F0/0
#ipaddress 192.168.6.1 255.255.255.0
#ip nat inside
#no shutdown
exit

#configure terminal
#interface F0/1
#ip address 10.10.10.1 255.255.255.0
#ip nat outside
#no shutdown
exit
//per farlo dinamico:
#ip nat pool servizi 10.10.10.1 
  10.10.10.1 prefix-length 24
#access-list 1 
  permit 192.168.6.0 0.0.0.255
#ip nat inside source 1 
  pool servizi overload
	\end{lstlisting}	

\end{multicols}
\end{document}