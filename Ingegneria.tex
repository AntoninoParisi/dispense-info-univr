\documentclass[a4paper, 10pt]{article}
\usepackage[italian]{babel}
\usepackage{sans}
\usepackage[T1]{fontenc}
\usepackage[utf8]{inputenc}
\usepackage{hyperref}
\hypersetup{
	hidelinks, 
	colorlinks = true,
	linkcolor = black
}

\title{Ingegneria del Software}
\date{}
\author{Matteo Danzi}
\begin{document}
	\maketitle
	
	\tableofcontents
	
	\newpage
	
	\section{Introduzione}
	L' ingegneria del software è lo \textit{studio delle metodologie, strumenti e teorie che
	stanno alla base della creazione dei software a livello professionale}.\\

	\noindent
	È lo studio a livello \textit{professionale} perché a livello \textit{artigianale} non sempre vi è la garanzia del prodotto, in quanto non è garantito che la soluzione sia \textbf{scalabile} (cioè
	permette di utilizzare il software per un ambito/problema più grande o differente rispetto a quello in cui è utilizzato).\\
	
	\noindent
	I prodotti software si suddividono in:
	\begin{itemize}
		\item \textbf{prodotti software generici} (e.g.: S.O., Applicazioni ...)
		
		prodotti in cui il progettista decide le caratteristiche del prodotto e deve accontentare le richieste da parte di molti utenti cercando di creare un progetto generico diretto a  una vasta categoria di persone.
		\item \textbf{prodotti software specifici/personalizzati}(e.g.: Applicativi per monitorare l'inquinamento)
		
		prodotti in cui l'utilizzatore richiede al progettista le caratteristiche che deve avere il prodotto, e il software è pesantemente influenzato da chi lo usa.
	\end{itemize}

	\noindent
	\textbf{Ma che cos'è il software?} 
	
	\noindent
	Il software non è altro che un insieme di programmi che svolgono un compito che viene opportunamente documentato, la cui documentazione indica com'è strutturato. \\
	
	\noindent
	\textbf{Caratteristiche di un buon prodotto software}:
	\begin{enumerate}
		\item \textbf{Mantenibilità}: il prodotto software deve durare e deve saper correggere
		eventuali errori, deve saper gestire il cambiamento (come, per esempio, l'esecuzione su diverse macchine).
		\item \textbf{Affidabilità}: un software deve eseguire quello che gli viene imposto senza
		compromettere la sicurezza degli utenti.
		\item \textbf{Usabilità}: quanto è facile da utilizzare un software.
	\end{enumerate}
	
	
	\noindent
	Quattro attività principali dell' ingegneria del software :
	\begin{enumerate}
		\item \textbf{\textit{Specifica del software}} : si considerano i requisiti di chi mi richiede di
		creare il software.
			\subitem{-} l'ingegneria dei requisiti considera i requisti dati e li trasforma in
			requisiti ad alto livello, ma piú rigorosi e meno ambigui.
		\item \textbf{\textit{Realizzazione del software}} : si appoggia sui requisiti e consiste di due fasi:
			\subitem{-} progettazione
			\subitem{-} implementazione
		\item \textbf{\textit{Validazione del software}} : è un fattore molto importante e consiste nella
		validazione da parte di un utente del software creato.
		\item \textbf{\textit{Evoluzione del software}} : consiste in modifiche e aggiornamenti apportati
		dopo la creazione finale del software; rappresenta il costo principale
		(più del 50\%!!!) di una vita media di un sw.
	\end{enumerate}
	
	
\end{document}