\documentclass[a4paper, 10pt]{article}
\usepackage[italian]{babel}
\usepackage{sans}
\usepackage[T1]{fontenc}
\usepackage[utf8]{inputenc}
\usepackage{hyperref}
\hypersetup{
	hidelinks, 
	colorlinks = true,
	linkcolor = black
}
\usepackage{geometry}
\geometry{a4paper, left=3cm,right=3cm,top=3cm,bottom=3cm}

\title{Ingegneria del Software}
\date{}
\author{Matteo Danzi \\Davide Bianchi}
\begin{document}
	\maketitle
	
	\tableofcontents
	
	\newpage
	
	\section{Introduzione}
	L' ingegneria del software è lo \textit{studio delle metodologie, strumenti e teorie che
		stanno alla base della creazione dei software a livello professionale}.\\
	
	\noindent
	È lo studio a livello \textit{professionale} perché a livello \textit{artigianale} non sempre vi è la garanzia del prodotto, in quanto non è garantito che la soluzione sia \textbf{scalabile} (cioè
	permette di utilizzare il software per un ambito/problema più grande o differente rispetto a quello in cui è utilizzato).\\
	I prodotti software si suddividono in:
	\begin{itemize}
		\item \textbf{prodotti software generici} (e.g.: S.O., Applicazioni ...)
		
		prodotti in cui il progettista decide le caratteristiche del prodotto e deve accontentare le richieste da parte di molti utenti cercando di creare un progetto generico diretto a  una vasta categoria di persone.
		\item \textbf{prodotti software specifici/personalizzati}(e.g.: Applicativi per monitorare l'inquinamento)
		
		prodotti in cui l'utilizzatore richiede al progettista le caratteristiche che deve avere il prodotto, e il software è pesantemente influenzato da chi lo usa.
	\end{itemize}
	
	\noindent
	\textbf{Ma che cos'è il software?} 
	
	\noindent
	Il software non è altro che un insieme di programmi che svolgono un compito qualsiasi e che viene opportunamente documentato, la cui documentazione indica com'è strutturato. \\
	
	\noindent
	\subsection*{Caratteristiche di un buon prodotto software}
	\begin{enumerate}
		\item \textbf{Mantenibilità}: 
		\begin{itemize}
			\item deve poter essere gestito su un periodo vita medio/lungo.
			\item deve saper gestire eventuali errori che non si erano trovati precedentemente.
			\item deve fare in modo che il software giri su piattaforme diverse.
		\end{itemize}
		\item \textbf{Affidabilità}: un software deve eseguire quello che gli viene imposto senza
		compromettere altri aspetti come ad esempio la sicurezza degli utenti.
		\item \textbf{Usabilità}: quanto è facile da utilizzare un software, saperlo utilizzare senza dover leggere le istruzioni. Ad esempio per utilizzare un browser web utilizzo sempre gli stessi criteri di usabilità e faccio in modo che sia utilizzabile in modo intuitivo.
	\end{enumerate}
	
	
	\noindent
	\subsection*{Quattro attività principali dell' ingegneria del software}
	\begin{enumerate}
		\item \textbf{\textit{Specifica del software}} : si considerano i requisiti di chi mi richiede di
		creare il software.
		\subitem{-} l'ingegneria dei requisiti considera i requisti dati e li trasforma in
		requisiti ad alto livello, ma più rigorosi e meno ambigui.
		\item \textbf{\textit{Realizzazione del software}} : si appoggia sui requisiti e consiste di due fasi:
		\subitem{-} progettazione
		\subitem{-} implementazione
		\item \textbf{\textit{Validazione del software}} : è un fattore molto importante e consiste nella
		validazione da parte di un utente del software creato.
		\item \textbf{\textit{Evoluzione del software}} : consiste in modifiche e aggiornamenti apportati
		dopo la creazione finale del software; rappresenta il costo principale
		(più del 50\%!!!) di una vita media di un sw.
		
		\subsection*{Problemi e sfde attuali dell' ingegneria del software}
		
		\begin{itemize}
			\item \underline{sicurezza}: privacy, frodi, danni a cose o persone ecc.
			\item \underline{costi}: si deve contenere i costi di produzione sotto controllo.
			\item \underline{etereogeneità}: sfruttare il sw su più macchine, soddisfare più richieste in
			contemporanea ecc.
			\item \underline{tempistiche}: sono sempre più ridotte.
		\end{itemize}
		
		\subsection*{Efficienza VS efficacia}
		L' \textit{efficienza} indica quanto materiale uso per risolvere un problema, è un'aspetto più computazionale, mentre l' \textit{efficacia} consiste nel chiedersi se un sw svolge il proprio compito per cui è stato creato, quanto un sw è in grado di risolvere il problema richiesto dall'utente e quindi se raggiunge l'obbiettivo delle specifiche.
		
		\noindent
		Tipologie software:
		\begin{enumerate}
			\item \underline{Sistemi Stand-Alone}: sistemi sw isolati a livello di rete, funzionano su una macchina isolata.
			\item \underline{Applicazioni interattive basate su transizioni}: scambio di dati concordato
			da entrambe le parti (vi è il concetto di "o tutto, o niente"). Sono per lo
			più basate sul web e quindi distribuite sulla rete. Ad esempio operazioni di scambio di dati tipo commercio elettronico(Amazon) oppure tutte le applicazioni bancarie.
			\item \underline{Sistemi embedded}: sistemi sw che sono innestati all'interno di apparecchi di varia natura (controllo treni, bancomat). Il software è a bordo di un sistema dedicato e quindi ho vincoli architetturali. 
			\item \underline{Applicazioni batch}: applicazioni che elaborano dati in grossi gruppi. Ad esempio stampare e produrre dati bancari di tutti i clienti di una banca in un certo istante oppure inserimento pesante di dati.
			\item \underline{Sistemi d'intrattenimento}: tipo videogiochi, film on demand, il daltonismo del Dalty ecc. Pesantemente basati sul web.
			\item \underline{Applicazioni di raccolta dati}: tipo Arpav, centraline per rilevare dati di inquinamento, i quali raccolti e 
			\item \underline{Sistemi di sistemi}: sistemi che si basano su altri sistemi (come Amazon,
			il quale fa scambio di merci e, contemporaneamente, raccoglie i dati e li
			rivende ad altre aziende).
		\end{enumerate}
	\end{enumerate}
		
		\section{Sviluppo agile}
		\subsection*{Introduzione}
		Nel mondo moderno, lo sviluppo rapido e la consegna in breve tempo di un prodotto finito sono i requisiti più importanti per i sistemi software.
		
		Il metodo di sviluppo agile si è sviluppato nei tardi anni 90 e il suo scopo principale era quello di ridurre il tempo di consegna di un prodotto software finito. 
		
		In un sistema agile di norma la progettazione e l'implementazione di un software sono sempre a stretto contatto, e le parti interessate al software sono spesso coinvolte nel processo di sviluppo, per scopi di specifica e validazione del prodotto stesso, che viene consegnato in più versioni (spesso con una documentazione minima, per focalizzare gli sforzi dello sviluppo nella scrittura di codice funzionante).
		
		Lo sviluppo agile del software ruota in generale attorno a 5 principi cardine:
		\begin{itemize}
			\item Coinvolgimento del cliente: sfruttato soprattutto per le sezioni di verifica del prodotto;
			
			\item Consegna incrementale: il cliente specifica i requisiti una porzione alle volta, e la loro realizzazione è inclusa nella consegna del prodotto più vicina;
			
			\item Sfruttamento delle abilità del team;
			
			\item Progettare basandosi sui cambiamenti dei requisiti, quindi sviluppare un prodotto che sia semplice da cambiare;
			
			\item Mantenere la semplicità del prodotto: se possibile lavorare anche attivamente per ridurre la complessità del software sviluppato finora.
		\end{itemize}
	
		\subsection*{Extreme programming}
		La tecnica di extreme programming estremizza le condizioni di lavoro per lo sviluppo agile del software, a volte anche facendo build del prodotto diverse volte al giorno, pubblicando release dopo brevissimi periodi di sviluppo e test (circa 2 settimane), e assicurandosi che la release venga pubblicata solo quando tutti i test siano passati.
		
		Le caratteristiche principali dell'extreme programming sono le seguenti:
		\begin{itemize}
			\item Planning dei task incrementale: le caratteristiche da includere nel software sono decise sulla base del tempo che richiedono per essere sviluppate, e sono divise in tasks;
			
			\item Piccole release: una versione con delle funzionalità di base è la prima ad essere rilasciata, successivamente vengono fatte release frequenti e con aggiunte di feature extra in maniera incrementale;
			
			\item Design semplice: il design è spesso progettato in maniera minimale, in modo da concentrare lo sviluppo per soddisfare al massimo i requisiti richiesti;
			
			\item Sviluppo del test necessario per testare una funzionalità prima di implementare la funzionalità stessa;
			
			\item Refactoring: tutti gli sviluppatori eseguono un refactoring del codice ogni volta che sono possibili delle ottimizzazioni, per mantenere il codice più semplice e mantenibile possibile;
			
			\item Pair programming: programmazione di coppia, per verificare ognuno il lavoro dell'altro;
			
			\item Proprietà collettiva: le coppie di sviluppatori lavorano su tutto il software, in modo da non creare zone di codice che solo una coppia sa interpretare e tutti si prendono la responsabilità di tutto il codice;
			
			\item Integrazione continua: ogni volta che una nuova funzionalità è implementata viene immediatamente integrata nel prodotto finito;
			
			\item Ritmo sostenibile: non sono permessi grandi overtime, per non abbassare la qualità del codice;
			
			\item Cliente sempre in loco: un rappresentante del cliente è sempre disponibile al team di sviluppo ed è responsabile per la consegna delle specifiche.
		\end{itemize}
		
		\paragraph{Refactoring} Il refactoring è una procedura (in alcuni punti anche automatizzata) per migliorare la qualità del codice e mantenerlo leggibile e facilmente modificabile, ad esempio riorganizzando la gerarchia delle classi per evitare codice duplicato, tenere un codice pulito per renderlo semplice da capire, usare delle librerie per evitare di sovraccaricare il codice che si scrivere con molte chiamate consecutive.
		
		\paragraph{Test-first} Il metodo test-first è basato sull'idea di sviluppare prima il test della feature che la feature stessa, in questo modo vengono a galla molti chiarimenti sulla funzionalità da implementare. Spesso i test di questo tipo sono vengono eseguiti facendo affidamento su un testing framework, ad esempio JUnit, che aiuta lo sviluppatore nella stesura e nell'esecuzione dei test stessi.
		
		\subsection*{Sistema Scrum}
		Il sistema \textit{Scrum} è un metodo di sviluppo basato più sul controllare lo sviluppo iterativo. È contruito su 3 fasi:
		\begin{itemize}
			\item Fase iniziale di pianificazione dove vengono specificati gli obiettivi del software e disegnata l'architettura del sistema;
			\item Fase di sviluppo, costituita da una serie di sprint iterativi che aggiungono sempre nuove feature al sistema;
			
			\item Al termine dello sviluppo, vengono scritti la documentazione e i manuali utente.
		\end{itemize}
		I vantaggi di un sistema Scrum sono dati dal fatto che il prodotto completo per essere sviluppato viene diviso in una serie di sotto-chunk, che vengono risolti man mano dagli sviluppatori durante ogni ciclo di lavoro.
		
		\subsection*{Problematiche dello sviluppo agile}
		Le più grandi problematiche dello sviluppo agile sono le seguenti:
		\begin{itemize}
			\item I metodi di lavoro agili sono più adatti alla scrittura di nuovo software piuttosto che alla manutenzione di software precedente, e la maggior parte delle grandi compagnie si affida di più a software datato che viene continuamente aggiornato;
			
			\item I metodi sono concepiti più per lo sviluppo con pochi sviluppatori concentrati in un solo luogo, non per grandi compagnie, per le quali potrebbero risultare dispersivi approcci di questo tipo;
			
			\item Il sistema agile spesso è in conflitto con quello che è l'approccio dell'azienda al sistema di sviluppo del software;
			
			\item Il sistema agile è concentrato maggiormente sulla frequenza delle release che non sulla documentazione, cosa che può rendere problematica la manutenzione di software scritto con questo sistema o software di grandi dimensioni. Per questo motivo si preferisce più che altro usare un approccio plan-driven e agile (una sorta di metodo ibrido), anche per adattarsi a progetti di software con un ciclo di vita lungo (che hanno quindi bisogno di documentazione) o di grandi dimesioni (in quanto richiedono un numero di sviluppatori che può essere sparso per il mondo).
		\end{itemize}
		
	\section{Ingegneria dei requisiti}
	Un requisito è una specifica che può andare dalla descrizione ad alto livello di un servizio fino ad un vincolo che il prodotto deve soddisfare.
	I requisiti si dividono in due macro-categorie:
	\begin{itemize}
		\item Requisiti utente: definiti in linguaggio naturale, specificano i servizi che il sistema deve fornire una volta sviluppato;
		
		\item Requisiti di sistema: sono indicati in un documento ben strutturato, che da una specifica dettagliata delle funzioni del sistema, dei servizi che offre e dei vincoli di funzionamento (di performance, ecc.).
	\end{itemize}

	\subsection*{Requisiti funzionali e non funzionali}
	I requisiti si possono suddividere inoltre in altre due grandi categorie:
	\begin{itemize}
		\item Requisiti funzionali: specificano quali servizi il sistema deve offrire, come il sistema dovrebbe reagire di fronte a determinati input, come dovrebbe comportarsi in determinate situazioni
		
		\item Requisiti non funzionali: costituiscono dei vincoli di sistema come ad esempio vincoli di tempo, vincoli di sviluppo ecc.
	\end{itemize}

 	È importante notare che i requisiti funzionali potrebbero anche essere ambigui o venire male interpretati, in tal caso è bene, prima di iniziare lo sviluppo, fare un controllo di completezza e consistenza dei requisiti forniti dall'utente, per evitare errori di progettazione del software.
 	
 	I requisiti funzionali invece non hanno bisogno di particolari tipi di analisi e sono classificati come segue:
 	\begin{itemize}
 		\item Requisiti del prodotto: specificano come il prodotto deve comportarsi;
 		
 		\item Requisiti organizzativi: requisiti derivati da politiche di organizzazione;
 		
 		\item Requisiti esterni: derivano da fattori esterni (legali, di interoperabilità...)
 	\end{itemize}
 
 	\subsection*{Processi di ingegneria dei requisiti}
 	Il processo di ingegneria dei requisiti è concentrato su 4 punti, svolti iterativamente:
 	\begin{enumerate}
 		\item Estrazione dei requisiti;
 		\item Analisi dei requisiti;
 		\item Validazione dei requisiti;
 		\item Gestione dei requisiti.
 	\end{enumerate}
	
	\subsubsection*{Estrazione dei requisiti}
	È il processo con il quale si determinano i requisiti da implementare:
	\begin{enumerate}
		\item Scoperta dei requisiti: richiede interazione diretta col cliente;
		\item Specifica dei requisiti: stesura dei documenti che specificano i requisiti di sistema e i requisiti utente, in modo che possano essere compresi anche dai clienti;
	\end{enumerate}

	\subsubsection*{Validazione dei requisiti}
	È la "dimostrazione" che i requisiti analizzati sono esattamente quello che il cliente vuole dal team di sviluppo. La sezione della validazione è importante, in quanto permette di rilevare gli errori nell'analisi dei requisiti che costerebbero molto tempo per essere risolti in fase di sviluppo.
	
	I punti da tenere sotto controllo sono i seguenti:
	\begin{enumerate}
		\item Validità: verifica che il sistema fornisca le funzioni che il cliente desidera;
		\item Consistenza: controllo di eventuali conflitti di requisiti;
		\item Completezza: verifica che tutte le funzioni desiderate dal cliente siano coperte;
		\item Realismo: verifica che i requisiti specificati siano realizzabili con il budget e le tecnologie disponibili;
		\item Verificabilità: descrizione di metodi per verificare i requisiti.
	\end{enumerate}

	\subsubsection*{Gestione dei requisiti}
	Questa parte del processo di ingegnerizzazione dei requisiti consiste nel definire delle politiche di gestione in caso di cambio di requisiti del sistema.
	Il processo di modifica dei requisiti passa attraverso un analisi dei costi di modifica del programma, un'analisi dei delle modifiche da apportare e poi viene avviata la modifica dell'implementazione del prodotto finito.
	
	
	
\end{document}