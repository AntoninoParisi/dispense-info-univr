\documentclass[a4paper, 11pt]{article}
\usepackage[english]{babel}
\usepackage[utf8]{inputenc}
\usepackage{amsmath}
\usepackage{graphicx}
\usepackage{float}
\usepackage{fixltx2e}
\usepackage{listings}
\usepackage{color}
\usepackage{latexsym}
\usepackage{lstautogobble}
\usepackage[colorinlistoftodos]{todonotes}
\usepackage[margin=3cm]{geometry}
\usepackage{hyperref}
\hypersetup{
	hidelinks, 
	colorlinks = true,
	linkcolor = black,
}

\newcommand{\od}{\textbf{output-deterministica }}
\newcommand{\revp}[1]{P_{#1}^{-1}}


\date{\today}

\newtheorem{definit}{Definizione}[subsection]
\newtheorem{thm}{Teorema}[subsection]

\begin{document}
 \clearpage
 \begin{titlepage}
 	\centering
 	\vspace*{\fill}
 	{\scshape\LARGE Università degli Studi di Verona \par}
 	\vspace{1.5cm}
 	\line(1,0){280} \\
 	{\huge\bfseries Sistemi ad eventi discreti\par}
 	\line(1,0){280} \\
 	\vspace{0.5cm}
 	{\scshape\Large Riassunto dei principali argomenti\par}
 	\vspace{2cm}
 	{\Large\itshape Giorgia Gulino, Davide Bianchi\par}
 	\vspace{1cm}

 	\vspace{5cm}
 	\vspace*{\fill}
 	% Bottom of the page
 	{\large \today\par}
 \end{titlepage}
 \thispagestyle{empty}
\newpage
\tableofcontents
\newpage

\section{Macchine a stati}

\begin{definit}[Macchina a stati deterministica]
Una macchina a stati è deterministica se valgono: \begin{itemize}
	\item esiste un solo uno stato iniziale;
	\item per ogni stato e per ogni input esiste solo un stato successivo;
\end{itemize}  
\end{definit}
Inoltre se $M_2$ è deterministica allora $M_1$ è \textbf{simulata} da $M_2$ sse è \textbf{equivalente} a $M_2$.

\subsection{Output-Determinismo}
\begin{definit}
	Una macchina a stati è \textbf{output-deterministica} se esiste un solo stato iniziale e per ogni stato e ogni coppia di I/O c'è un solo stato successivo. Se $M_2$ è \od allora $M_2$ \textbf{simula} $M_1$.
\end{definit}

\begin{center}
\textbf{determinismo} $\Rightarrow$ \textbf{output-determinismo}, non vale il viceversa.
\end{center}

\subsection{Non-Determinismo}
In una macchina a stati non deterministica può esistere più di uno stato iniziale e per ogni stato e ogni coppia di I/O può esistere più di uno stato successivo. Se $M_2$ è non deterministica, $M_1$ è \textbf{simulata} da $M_2$ allora $M_1$ \textbf{raffina} $M_2$, ma non viceversa.


Una macchina a stati è progressiva quando l'evoluzione è definita per ogni ingresso, cioè la funzione è definita come \[States \times Inputs \Rightarrow \mathcal{P}(States \times Outputs) \setminus \emptyset \] dove $\mathcal{P}$ rappresenta l'insieme potenza e l'insieme vuoto impone che sia progressiva.

\subsection{Equivalenze}
Data una macchina a stati $X$:
\begin{itemize}
\item se $X$ è deterministica: $input[M_1]=input[M_2]; output[M_1]=output[M_2]$.
\item se $X$ è non deterministica: $Behaviour[M_1]=Behaviour[M_2]$.
\item se due macchine a stati $M_1$ e $M_2$ sono bisimili, allora sono equivalenti.
\end{itemize}

\begin{definit}[Raffinamento]
	$M_1$ \textbf{raffina} $M_2 \Leftrightarrow  \Big(Inputs[M_1]=Inputs[M_2]\ \wedge\ Outputs[M_1]=Outputs[M_2] \wedge Behaviour[M_1] \subseteq Behaviour[M_2]\Big)$.
\end{definit}


\subsection{Bisimulazione}
Bisimulazione tra $M_1$ e $M_2$ sse l'unione delle \textbf{simulazioni} è simmetrica e c'è \textbf{isomorfismo} tra minimize($M_1$) e minimize($M_2$).

\subsection{Rel. RAFFINAMENTO/SIMULAZIONE AFSND $\rightarrow$ AFSD} Se $M_1$ è det, $M_1$ è \textbf{simulata} da $M_2$ sse $M_1$ è equivalente a $M_2$, cioè se $M_1$ raffina $M_2$ e viceversa.
\subsection{Rel. RAFFINAMENTO/SIMULAZIONE AFSND $\rightarrow$ AFSpseudo-nondet} Se $M_2$ è psuedo-non det, $M_1$ è \textbf{simulata} da $M_2$ sse $M_1$ è equivalente a $M_2$, cioè se $M_1$ \textbf{raffina} $M_2$.
\subsection{Rel. RAFFINAMENTO/SIMULAZIONE AFSND $\rightarrow$ AFSND} Se $M_2$ non è deterministica, $M_1$ è \textbf{simulata} da $M_2$ implica $M_1$ \textbf{raffina} $M_2$, ma $M_1$ raffina $M_2$ non implica $M_1$ \textbf{simula} $M_2$.
\subsection{Simulazione per Det $\rightarrow$ $M_1$ da $M_2$}
\begin{itemize}
\item $\forall$ p $\in$  PossibiliInitialState[$M_1$], $\exists$ q $\in$ PossibiliInitialState[$M_2$], (p,q) $\in$ S.
\item $\forall$ p $\in$  Stati[$M_1$], $\forall$ q $\in$ Stati[$M_2$].
\begin{itemize}
\item if $(p,q) \in S \Rightarrow \forall x \in Input, \forall y \in Output, \forall p_1 \in  Stati[M_1]$;
\item if (p1,y) $\in$ PossibiliUpdates[$M_1$](p,x) $\Rightarrow$ $\exists$ q1 $\in$ Stati[$M_1$], (q1,y) $\in$ PossibiliUpdates[$M_2$](q,x) e (p1,q1) $\in$ S. (S contiene coppie di stati iniziali e coppie consultanti l'algoritmo).
\item $\forall$ p $\in$ Stati[$M_1$] $\exists$ q $\in$ Stati[$M_2$] per cui $\forall$ I/O possibili c'è corrispondenza tra I/O uguali di p e (p,q) $\in$ S. 
\end{itemize}
\end{itemize}
\subsection{Simulazione per Output-Det}Data M ASFND trova la macchina output-det det(M) equivalente a M.\\ SUBSET CONSTRUCTION 
\begin{itemize}
\item InitialState[det(M)] = PossibileInitialState[M]
\item States[det(M)]={InitialState[det(M)]}
\item Ripeti finché nuove transizioni possono essere aggiunte a det(M). Scegli
\begin{itemize}
\item P $\in$ States[det(M)] e (x,y) $\in$ Input x Output
\item Q = {q $\in$ States[M] | $\exists$ p $\in$ P, (q,y) $\in$ PossibleUpdates[M](p,x)}
Se Q $\ne$ 0 allora States[det(M)]= States[det(M)] $\cup$ {Q}\\
Update[det(M)](p,x)=(q,y)\\
Raggruppa tutti gli stati iniziali, $\forall$ coppia I/O raggruppa tutti gli stati per cui quest'ultima è Possibleupdate.
\end{itemize}
\end{itemize}
\subsection{Simulazione}
\begin{itemize}
\item Se p $\in$ PossibleInitialState[$M_1$] e PossibleInitialState[m2] = {q} $\Rightarrow$ (p,q) $\in$ S.
\item Se (p,q) $\in$ S e (p1,y) $\in$ PossibleUpdates[$M_1$](p,x) e PossibleUpdates[$M_2$](q,x) = {q}. 
\end{itemize}
\subsection{Bisimulazione per Det} Una relazione binaria B è una \emph{bisimulazione} sse:
\begin{itemize}
\item InitialState[$M_1$], InitialState[$M_2$] $\in$ B 
\item $\forall$ p $\in$ Stati[$M_1$], $\forall$ q $\in$ Stati[$M_2$]:
\begin{itemize}
\item if (p,q) $\in$ B $\Rightarrow$ $\forall$ x $\in$ Input[$M_1$], Output[$M_1$](p,x) = Output[$M_2$](q,x)\\
(nextState[$M_1$](p,x),nextState[$M_2$](q,x)) $\in$ B.
Stati iniziali di $M_1$ e $M_2$ sono in relazione e ogni coppia (p,q) relazionati, $\forall$ input producono lo stesso output e nextState Relazionati.
\end{itemize}
\end{itemize}

\section{Linguaggi}
\subsection{Introduzione}

\begin{definit}[Linguaggio]
	Dato un insieme di simboli $E$, un traccia (ovvero una sequenza finita di simboli di $E$), definiamo linguaggio un qualsiasi sottoinsieme $L \subseteq E^\ast$.
\end{definit}

Definiamo inoltre l'insieme dei prefissi di un linguaggio come l'insieme \[ \overline{L} := \lbrace s \in E^\ast : (\exists t \in E^\ast) st \in L \rbrace \]

\subsection{Automi}

In parole povere, un automa è una struttura matematica che genera un determinato linguaggio.
\begin{definit}[Automa]
	Un automa è un tupla di sei elementi \[ G = (X,E,f,\Gamma,x_0,X_m) \] dove: \begin{itemize}
		\item $X$ è l'insieme degli stati;
		\item $E$ è l'insieme di eventi associati alle transizioni in $G$;
		\item $f:X \times E \to X$ è la \textbf{funzione di transizione}; dicendo che $f(x,e) = y$ si sta dicendo che esiste una transizione etichettata con $e$ che porta dallo stato $x$ allo stato $y$;
		\item $\Gamma: X \to 2^E$ è la \textbf{funzione degli eventi attivi}; $\Gamma(x)$ indica l'insieme degli eventi per i quali $f(x,e)$ è definita;
		\item $x_0$ è lo stato iniziale;
		\item $X_m \subseteq X$ è l'insieme degli \textbf{stati marcati}.
	\end{itemize}
\end{definit}

\begin{definit}[Linguaggio generato]
	Il linguaggio generato da un automa è definito come: \[ L(G) = \lbrace s \in E^\ast : f(x_0,s) \text{ è definita} \rbrace \]
\end{definit}

\begin{definit}[Linguaggio marcato]
	Il linguaggio marcato da un automa è definito come: \[ L_m(G) = \lbrace s \in L(G) : f(x_0,s) \in X_m \rbrace \]
\end{definit}

\begin{definit}[Equivalenza tra automi]
	Due automi $G_1$ e $G_2$ si dicono equivalenti se vale che \[ L(G_1) = L(G_2) \wedge L_m(G_1) = L_m(G_2) \]
\end{definit}

\subsubsection{Composizione di automi}
\paragraph{Proiezioni naturali.}
Le proiezioni naturali sono funzioni del tipo $$P_i : (E_1 \cup E_2)^\ast \to E_i^\ast$$ definite come 
\begin{align*}
	P_i(\epsilon) &= \epsilon \\
	P_i(\sigma) &= \begin{cases}
	\sigma &\text{se } \sigma \in E_i \\
	\epsilon &\text{se } \sigma \notin E_i
	\end{cases} \\
	P_i(s\sigma) &= P_i(s)P_i(\sigma) \text{ per } s \in (E_1 \cup E_2)^\ast, \sigma \in E_1 \cup E_2
\end{align*}

Delle proiezioni naturali esistono anche le inverse: come le proiezioni sono estese ai linguaggi come
\[
	P_i(L) = \lbrace t \in E_i^\ast: \exists s \in L(P_i(s) = t) \rbrace \text{ con } L \subseteq (E_1 \cup E_2)^\ast
\]
anche le proiezioni inverse sono estese come
\[
	\revp{i}(L_i) = \lbrace s \in (E_1 \cup E_2)^\ast: \exists t \in L_i(P_i(s) = t) \rbrace
\]
per $L_i \subseteq E_i^\ast$.

\paragraph{Prodotto di automi.} Il prodotto di due automi $G_1 = (X_1, E_1, f_1, \Gamma_1, x_{01}, X_{m1})$ e $G_2 = (X_2, E_2, f_2, \Gamma_2, x_{02}, X_{m2})$ è l'automa risultato \[ G_1 \times G_2 = (X_1 \times X_2, E_1 \cap E_2,f,\Gamma_{1 \times 2}, (x_{01}, x_{02}), X_{m1} \times X_{m2}) \] dove 
\begin{align*} f((x_1, x_2), \sigma) &= \begin{cases}
(f(x_1, \sigma), f_2(x_2, \sigma)) &\text{se } \sigma \in \Gamma_1(x_1) \cap \Gamma_2(x_2) \\
\text{indefinita} &\text{altrimenti}
\end{cases}  \\
\Gamma_{1 \times 2}(x_1, x_2) &= \Gamma_1(x_1) \cap \Gamma_2(x_2)
\end{align*}

Le proprietà di questa composizione sono le seguenti: \begin{enumerate}
	\item $L(G_1 \times G_2) = L(G_1) \cap L(G_2)$
	\item $L_m(G_1 \times G_2) = L_m(G_1) \cap L_m(G_2)$
\end{enumerate}

\paragraph{Composizione parallela di automi.} La composizione parallela di due automi $G_1 = (X_1, E_1, f_1, \Gamma_1, x_{01}, X_{m1})$ e $G_2 = (X_2, E_2, f_2, \Gamma_2, x_{02}, X_{m2})$ è l'automa risultato \[ G_1 \vert\vert G_2 = (X_1 \times X_2, E_1 \cup E_2,f,\Gamma_{1 \vert \vert 2}, (x_{01}, x_{02}), X_{m1} \times X_{m2}) \]
dove \[ 
	f((x_1, x_2), \sigma) = \begin{cases}
	(f_1(x_1, \sigma), f_2(x_2, \sigma)) &\text{se } \sigma \in \Gamma_1(x_1) \cap \Gamma_2(x_2) \\
	(f_1(x_1, \sigma), x_2) &\text{se } \sigma \in \Gamma_1(x_1) \setminus E_2 \\
	(x_1, f(x_2, \sigma)) &\text{se } \sigma \in \Gamma_2(x_2) \setminus E_1 \\
	\text{indefinita} &\text{altrimenti}
	\end{cases}
 \]
 
La proiezione parallela gode delle seguenti proprietà:\begin{enumerate}
	\item $L(G_1 \vert \vert G_2) = \revp{1}\left[ L(G_1) \right] \cap \revp{2}\left[ L(G_2) \right] $
	\item $L_m(G_1 \vert \vert G_2) = \revp{1}\left[ L_m(G_1) \right] \cap \revp{2}\left[ L_m(G_2) \right] $
	\item $G_1 \vert \vert G_2 = G_2 \vert \vert G_1$
\end{enumerate}

\subsection{Controllabilità e osservabilità di un linguaggio}

Diamo la definizione formale di controllabilità, poi passeremo ad una descrizione più semplice.
\begin{definit}[Controllabilità]
	Siano $K$ e $M = \overline{M}$ linguaggi dell'alfabeto di eventi $E$, con $E_{uc} \subseteq E$. Si dice che $K$ è controllabile rispetto a $M$ e $E_{uc}$ se per tutte le stringhe $s \in \overline{K}$ e per tutti gli eventi $\sigma \in E_{uc}$ vale \[s \sigma \in M \Rightarrow s\sigma \in \overline{K}\footnotemark \]
\end{definit}
\footnotetext{Equivalente a $\overline{K}E_{uc} \cap M \subseteq\overline{K}$.}

La concezione di controllabilità si rende necessaria in quanto un evento incontrollabile può portare ad un crash o ad un fallimento del sistema. 

Supponiamo che $E = E_c \cup E_{uc}$ dove: \begin{itemize}
	\item $E_c$ è l'insieme di eventi controllabili;
	\item $E_{uc}$ è l'insieme di eventi non controllabili.
\end{itemize}

Supponiamo inoltre che la funzione di transizione di un automa $G$ possa essere controllata da un agente esterno, che abbia la capacità di disabilitare gli eventi incontrollabili.




\begin{definit}[Osservabilità]
Si considerino i linguaggi K e M = $\overline{M}$ definiti sull'alfabeto di eventi E, con $E_{c} \subseteq E, E_o \subseteq E$ e $P$ la proiezione naturale da $E^\ast \Rightarrow E_0^\ast$ .\\ Si dice che $K$ è osservabile rispetto a $M, E_{o},E_{c}$ se per tutte le stringhe s $\in$ $\overline{K}$ e per tutti gli eventi $\sigma$ $\in$ $E_c$ abbiamo: 
\begin{center}(s$\sigma$ $\notin$ K) $\land$ (s$\sigma$ $\in$ M) $\Rightarrow$ $P^{-1}$[P(s)] $\sigma$  $\cap$ $\overline{K}$ = $\emptyset$
\end{center}
\end{definit}
L'insieme di stringhe denotato dal termine $P^{-1}$[P(s)] $\sigma$  $\cap$ $\overline{K}$ contiene tutte le stringhe che hanno la medesima proiezione di s e possono essere prolungate in K con il simbolo $\sigma$. SE tale insieme non è vuoto, allora K contiene due stringhe s e s' tali che P(s)=P(s') per cui s$\sigma$ $\notin$ $\overline{K}$ e s'$\sigma$ $\in$ $\overline{K}$. Tali due stringhe richiederebbero un'azione di controllo diversa rispetto a $\sigma$ (disabilitare $\sigma$ nel caso di s, abilitare $\sigma$ nel caso di s'), ma un supervisore non saprebbe distinguere tra s e s' per l'osservabilità ristretta. Non potrebbe quindi esistere un supervisore che ottiene esattamente il linguaggio $\overline{K}$.

\subsection{Proprietà di Controllabilità}
Esistono due tipi di linguaggi derivati da K:
\begin{itemize}
\item K\textsuperscript{$\uparrow$C} il \textit{il sottolinguaggio supremo di K}
\item K\textsuperscript{$\downarrow$C} il \textit{il sovralinguaggio controllabile infimo di K}
\end{itemize}
Abbiamo i seguenti rapporti: 
\begin{center}
$\emptyset$ $\subseteq$ K\textsuperscript{$\uparrow$C} $\subseteq$ K $\subseteq$ $\overline{K}$ $\subseteq$ K\textsuperscript{$\downarrow$C} $\subseteq$ M
\end{center}
\begin{itemize}
\item Se K\textsubscript{1} e K\textsubscript{2} sono controllabili, allora K\textsubscript{1} $\cup$ K\textsubscript{2} è controllabile. 
\item Se K\textsubscript{1} e K\textsubscript{2} sono controllabili, allora K\textsubscript{1} $\cap$ K\textsubscript{2} non ha bisogno di essere controllabile.
\item Se K\textsubscript{1} e K\textsubscript{2} sono non in conflitto ed entrambi controllabili, allora K\textsubscript{1} $\cap$ K\textsubscript{2} è controllabile. \textbf{Si ricorda che K\textsubscript{1} e K\textsubscript{2} si dicono non in conflitto qualora $\overline{K}$\textsubscript{1} $\cap \overline{K}$\textsubscript{2} = $\overline{ K_1 \cap K_2 }$ }
\item Se K\textsubscript{1} e K\textsubscript{2} sono prefisso-chiuso e controllabili, allora $K_1 \cap K_2$ è prefisso-chiuso e controllabile.
\end{itemize}
Definiamo due classi di linguaggi: 
\begin{center}
% \textit{C}\textsubscript{in}(K) := {L $\subseteq$ K : $\overline{L}$E_{uc} $\cap$ M $\subseteq$ $\overline{L}$}
\end{center}
\begin{center}
% \textit{CC}\textsubscript{out}(K) := { L $\subseteq$ E* : (K $\subseteq$ L $\subseteq$ M) $\land$ ( $\overline{L}$ = L ) $\land$ ( $\overline{L}$E_{uc} $\cap$ M $\subseteq$ $\overline{L}$ ) }
\end{center}
\subsection{Riguardo il sottolinguaggio supremo}
\begin{itemize}
\item \textit{C}\textsubscript{in}(K) è un insieme parzialmente ordinato (\textit{o poset}) che è chiuso sotto unioni arbitrarie.
\item \textit{C}\textsubscript{in}(K) possiede un unico elemento \textit{supremo}. Definito come:
\begin{center}
\[ {K}^{\uparrow C} :=   \bigcup_{L \in \textit{C}\textsubscript{in}(K)} L \]
\end{center}
che è un elemento ben-definito di \textit{C}\textsubscript{in}(K).
\item K\textsuperscript{$\uparrow$C} è chiamato \textit{sottolinguaggio supremo controllabile di K.}
\begin{itemize}
\item Nel caso peggiore, K\textsuperscript{$\uparrow$C} = $\emptyset$ , dal momento che $\emptyset$ $\in$ \textit{C\textsubscript{in}(K)}
\item Se K è controllabile, allora K\textsuperscript{$\uparrow$C} = K 
\item Osserviamo che K\textsuperscript{$\uparrow$C} non necessita di essere prefisso-chiuso in generale
\end{itemize}
\end{itemize}
\subsection{Riguardo il sovralinguaggio infimo}
\begin{itemize}
\item \textit{CC}\textsubscript{out}(K) è un(\textit{ poset}) che è chiuso sotto intersezioni arbitrarie (e unioni).
\item \textit{CC}\textsubscript{out}(K) possiede un unico elemento \textit{infimo}. Definito come:
\begin{center}
\[ {K}^{\downarrow C} :=   \bigcap_{L \in \textit{CC}\textsubscript{out}(K)} L \]
\end{center}
che è un elemento ben-definito di \textit{CC}\textsubscript{out}(K).
\item Chiamiamo K\textsuperscript{$\downarrow$C} il sovralinguaggio infimo a prefisso-chiuso e controllabile di K.
\begin{itemize}
\item Nel caso peggiore, K\textsuperscript{$\downarrow$C} = M, dal momento che M $\in$ \textit{CC\textsubscript{out}}(K).
\item Se K è controllabile, allora K\textsuperscript{$\downarrow$C} = $\overline{K}$.
\end{itemize}
\end{itemize}




\end{document} 
