\documentclass[a4paper, 11pt]{article}
\usepackage[english]{babel}
\usepackage[utf8]{inputenc}
\usepackage{amsmath}
\usepackage{graphicx}
\usepackage{float}
\usepackage{fixltx2e}
\usepackage{listings}
\usepackage{color}
\usepackage{latexsym}
\usepackage{lstautogobble}
\usepackage[colorinlistoftodos]{todonotes}
\usepackage[margin=3cm]{geometry}
\usepackage{hyperref}
\usepackage{libertine}
\usepackage{tikz}
\hypersetup{
	hidelinks, 
	colorlinks = true,
	linkcolor = black,
}

\usetikzlibrary{shapes, arrows}

\newtheorem{definit}{Definizione}[subsection]

\begin{document}
	\clearpage
	\begin{titlepage}
		\centering
		\vspace*{\fill}
		{\scshape\LARGE Università degli Studi di Verona \par}
		\vspace{1.5cm}
		\line(1,0){280} \\
		{\huge\bfseries Sistemi informativi\par}
		\line(1,0){280} \\
		\vspace{0.5cm}
		{\scshape\Large Riassunto dei principali argomenti\par}
		\vspace{2cm}
		{\Large\itshape Davide Bianchi\par}
		\vspace{1cm}
		
		\vspace{5cm}
		\vspace*{\fill}
		% Bottom of the page
		{\large \today\par}
	\end{titlepage}
	\thispagestyle{empty}
	\newpage
	\tableofcontents
	\newpage
	
	
	\section{Teoria dell'organizzazione}
	\subsection{Introduzione}
	Iniziamo con alcune definizioni \textit{estremamente} tediose.
	\begin{definit}[Sistema informativo]
		Il Sistema Informativo (SI) è la componente (sottosistema) di una organizzazione che gestisce le informazioni di interesse.
	\end{definit}
	
	\begin{definit}[Organizzazione]
		Un’organizzazione è : \begin{itemize}
			\item il processo attraverso il quale tale insieme di persone viene strutturato secondo i principi di divisione del lavoro e coordinamento;
			\item il risultato del processo di divisione del lavoro e coordinamento.
		\end{itemize}
	\end{definit}
	
	\begin{definit}[Azienda]
		Un'azienda, nell'economia aziendale, è un'organizzazione di uomini e mezzi finalizzata alla soddisfazione di bisogni umani attraverso la produzione, la distribuzione o il consumo di beni economici.
	\end{definit}
	
	\begin{definit}[Organizzazione aziendale]
		Il processo attraverso il quale l'insieme di persone che partecipano direttamente allo svolgimento dell'attività dell'azienda viene strutturato secondo i principi di divisione del lavoro e coordinamento.
	\end{definit}
	L'organizzazione aziendale ha sempre almeno i macro processi operativo e gestionale, e dispone di risorse materiali, umane e informative.
	
	\begin{definit}[Tecnologie informatiche]
		Insieme di sistemi, strumenti e tecniche predisposti per automatizzare il trattamento delle informazioni.
	\end{definit}
	
	Un sistema informativo aziendale è una collezione di elementi interconnessi che gesticono la raccolta, l'elaborazione e la restituzione di informazioni. \\
	
	Un sistema produttivo aziendale è basato su \textit{obiettivi} (output atteso), \textit{input} ed \textit{output} effettivi. Definiamo inoltre i concetti di \textit{efficienza}, ovvero il costo di raggiungimento degli obiettivi, e di \textit{efficacia}, ovvero il grado di raggiungimento degli obiettivi. A grandi linee: \[ \text{\textit{Efficienza}} = \frac{Output}{Input} \qquad \text{\textit{Efficacia}} = \frac{Output}{Obiettivi}  \]
	
	Efficienza ed efficacia subiscono impatti differenti rispetto ad una innovazione delle risorse tecnologiche. Nel caso dell'efficienza: \begin{itemize}
		\item Riduzione dei costi unitari;
		\item Aumento di produzione a parità di risorse;
		\item Incentivo alla crescita delle dimensioni organizzative;
		\item Maggiore complessità strutturale;
		\item Cambiamenti della struttura organizzativa.
	\end{itemize}
	
	Nel caso dell'efficacia: \begin{itemize}
		\item Più efficiente uso dei fattori produttivi a parità di volumi di produzione (economie di scopo);
		\item Razionalizzazione dell'uso di risorse;
		\item Maggiore efficienza spesso legata a differenziazione dei prodotti e all’ampliamento della gamma.
	\end{itemize}
	
	L'organizzazione è un sistema aperto, influenzato da variabili ambientali, che vengono riassunte nell'\textit{incertezza ambientale}.
	L'incertezza ambientale determina i requisiti di capacità elaborativa della organizzazioni e l'adeguatezza del sistema informativo.
	
	\begin{definit}[Capacità elaborativa]
		Adeguatezza di un’organizzazione rispetto alle necessità di elaborare informazioni a essa imposte dai propri obiettivi e dal contesto in cui opera.
	\end{definit}
	
	L'ambiente è riassunto nel modello della piramide di Anthony, ovvero una piramide divisa in livelli gerarchici.
	Il layer più alto si occupa delle decisioni strategiche, quello centrale delle decisioni direzionali e quello basso di quelle operative.
	% FIGURA DELLA PIRAMIDE DI ANTHONY
	
	
	
	\subsection{Informazione come risorsa organizzativa}
	\paragraph{Caratteri principali.} L'informazione è la vera risorsa nelle attività organizzative, infatti viene scambiata ed elaborata; è immateriale, non è facilmente divisibile, può essere soggetta ad obsolescenza e si autorigenera.
	
	La capacità autorigenerativa dell'informazione permette di instaurare circoli virtuosi di generatori di conoscenza e di arricchimento delle informazioni disponibili, che si traducono in un incremento dei processi produttivi.
	
	\paragraph{Overload e underload informativo.} L'\textit{overload informativo} è un aumento incontrollato dell'informazione disponibile, che eccede la capacità di elaborazione individuale, con conseguente rallentamento nell'elaborazione. L'\textit{underload informativo} è invece una disponibilità di informazione al di sotto delle capacità individuali, con conseguente presa di decisioni in tempi brevi.
	
	\subsection{Sistemi informativi verticali e orizzontali}
	I sistemi informativi possono essere immaginati a due versi. 
	
	I sistemi informativi verticali sono stati i primi ad essere supportati dai sistemi informatici, tuttavia al crescere dell'incertezza i vertici sono sovraccaricati dai compiti decisionali. 
	
	I sistemi informativi orizzontali invece sono costruiti sulla delega delle decisioni e sui collegamenti sullo stesso layer che aumentano la capacità elaborativa (team di lavoro, task force).
	
	\section{Classificazione dei sistemi informativi}
	Vi sono varie possibili classificazioni dei sistemi informativi, ovvero: \begin{itemize}
		\item tipologie di SI disposti lungo la piramide aziendale (definizioni e funzioni attribuite a seconda del loro livello nella piramide);
		\item tipologie di SI diposti nelle varie aree gestionali dell'impresa.
	\end{itemize}

	\subsection{SI disposti lungo la piramide aziendale}
	I SI disposti lungo la piramide di Anthony sono i seguenti:
	\begin{enumerate}
		\item \textit{Transaction Processing Systems (TPS)}: gestione delle transazioni, quali ordini ecc. Sono alla base della piramide;
		
		\item \textit{Management Information Systems (MIS)}: sono al livello immediatamente sopra ai TPS e rappresentano periodicamente le informazioni raccolte dai TPS. Sono alla base del sistema di reportistica delle aziende;
		
		\item \textit{Decision Support Systems (DSS)}: affiancano il management delle decisioni di non routine e permettono di simulare ipotesi per verificare la validità di una gestione.
		
		\item \textit{Executive Information Systems (EIS)}: sono al vertice della gerarchia, aiutano i senior manager alla gestione.
	\end{enumerate}
	
	\subsection{Portafoglio applicativo}
	Il portafoglio applicativo è l'insieme delle applicazioni utili in azienda. È diviso in 3 segmenti principali: \begin{itemize}
		\item Portafoglio direzionale: insieme delle applicazioni informatiche a supporto dei cicli di pianificazione strategica;
		
		\item Portafoglio istituzionale: applicazioni informatiche per i processi di supporto all’amministrazione;
		
		\item Portafoglio operativo: applicazioni informatiche per i processi primari dell’azienda.
	\end{itemize}
	
	
	
	
	
	
	
	
	
	
\end{document}