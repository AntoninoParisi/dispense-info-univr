\documentclass[a4paper, 10pt]{article}

\usepackage[italian]{babel}
\usepackage[T1]{fontenc}
\usepackage[utf8]{inputenc}
\usepackage{amsmath}
\usepackage{amsfonts}
\usepackage{amsthm}
\usepackage{xfrac}
\usepackage{semantic}
\usepackage{titling}
\usepackage[a4paper, landscape,left=1.5cm, right=1.5cm, top=1cm, bottom=1cm]{geometry}
\usepackage{listing}
\usepackage{stmaryrd}
\usepackage{multicol}
\usepackage{lipsum}
\setlength{\columnseprule}{0.4pt}
\newcommand{\numberset}{\mathbb}
\newcommand{\Z}{\numberset{Z}}
\newcommand{\integer}{\textup{int}}
\newcommand{\bool}{\textup{bool}}
\newcommand{\skipp}{\textup{skip}}
\newtheorem{thm}{Teorema}[]
\newtheorem{lemma}{Lemma}
\theoremstyle{definition}
\newtheorem{prf}{Proof}[]
\newtheorem{base}{Caso Base}
\newtheorem{ind}{Caso induttivo}
\pagenumbering{gobble}
\setlength{\droptitle}{-5cm}
\date{}
\newcommand{\infer}[4]{\inference[\textbf{#1}]{#2}{#3}#4 }
\newcommand{\brule}[2]{\langle #1 \rangle \Downarrow #2}
\newcommand{\bbrule}[2]{ #1 \Downarrow #2}
\newcommand{\srule}[2]{\langle #1 \rangle \rightarrowtriangle \langle #2 \rangle}
\newcommand{\srulea}[2]{ #1  \rightarrowtriangle #2}
\newcommand{\ssrule}[2]{\langle #1 \rangle \rightarrowtriangle^{*} \langle #2 \rangle}
\newcommand{\memrep}[3]{#1 \lbrack #2 \mapsto #3 \rbrack}
\newcommand{\code}[1]{\textup{\lstinline{#1}}}
\newcommand{\ifthenelse}[3]{\text{if } #1 \text{ then } #2 \text{ else } #3}
\newcommand{\whiledo}[2]{\text{while } #1 \text{ do } #2}
\newcommand{\doreturn}[2]{\text{do } #1 \text{ return } #2}
\newcommand{\awaitprotectend}[2]{\text{await } #1 \text{ protect } #2 \text{ end}}
\newcommand{\letin}[2]{\text{let } #1 \text{ in } #2}
\newcommand{\subs}[3]{#1 \lbrace \sfrac{#2}{#3} \rbrace}
\newcommand{\goesto}{\rightarrowtriangle}
\title{\textbf{Regole di inferenza}}


\usepackage[OT1]{eulervm}

\begin{document}

	\maketitle
	\section*{Semantica big-step}
	\begin{align*}
		&\infer{(B-Num)}{-}{\brule{n,s}{n}}{}	
		&\infer{(B-Loc)}{-}{\brule{l,s}{s(l)}}{}\qquad\qquad
		&\infer{(B-Skip)}{-}{\brule{\skipp,s}{s}}{} 
		&\infer{(B-Add)}{\brule{E_1,s}{n_1} \quad \brule{E_2,s}{n_2}}{\brule{E_1+E_2,s}{n_3}}{n_3 = add(n_1, n_2)}
	\end{align*}
	\begin{align*}
		&\infer{(B-Assign)}{\brule{E,s}{n}}{\brule{l:=e,s}{\memrep{s}{l}{n}}}{}  &\infer{(B-Assign.s)}{\brule{E,s}{n}}{\brule{l:=e,s}{\langle \skipp, \memrep{s}{l}{n}} \rangle}{} \qquad\qquad
		&\infer{(B-Seq)}{\brule{C_1,s}{s_1} \quad \brule{C_2,s_1}{s'} }{\brule{C_1;C_2,s}{s'}}{} 
		&\infer{(B-Seq.s)}{\brule{C_1,s}{\langle \skipp,s_1 \rangle} \quad \brule{C_2,s_1}{\langle r,s' \rangle} }{\brule{C_1;C_2,s}{\langle r,s' \rangle}}{}
	\end{align*}
	\begin{align*}
		&\infer{(B-If.T)}{\brule{B,s}{true} \quad \brule{C_1,s}{s'}}{\brule{\ifthenelse{B}{C_1}{C_2},s}{s'}}{} 
		&\infer{(B-If.T)}{\brule{B,s}{true} \quad \brule{C_1,s}{\langle r,
				s'\rangle}}{\brule{\ifthenelse{B}{C_1}{C_2},s}{\langle r, s'\rangle}}{} \qquad
		&\infer{(B-If.F)}{\brule{B,s}{false} \quad \brule{C_2,s}{s'}}{\brule{\ifthenelse{B}{C_1}{C_2},s}{s'}}{}
		&\infer{(B-If.F)}{\brule{B,s}{false} \quad \brule{C_2,s}{\langle r, s'\rangle}}{\brule{\ifthenelse{B}{C_1}{C_2},s}{\langle r, s\rangle}}{}
	\end{align*}
	\begin{align*}
		&\infer{(B-While.T)}{\brule{B, s}{true}\quad \brule{C, s}{s_1}\quad \brule{\whiledo{B}{C}, s_1}{s'}}{\brule{\whiledo{B}{C}, s}{s'}}{}
		&\infer{(B-While.F)}{\brule{B, s}{false}}{\brule{\whiledo{B}{C}, s}{s}}{} \qquad
		&\infer{(B-Do)}{\brule{B,s}{v_1}\quad \brule{C,s}{v_2}}{\brule{\doreturn{E}{C}, s}{v_2}}{}
	\end{align*}
	\begin{align*}
		&\infer{(B-While.UN)}{\brule{\ifthenelse{B}{(C;\whiledo{B}{C})}{\skipp}, s}{r, s'}}{\brule{\whiledo{B}{C}, s}{r, s'}}{}
		&\infer{(B-Await)}{\brule{B, s}{true, s_1}\quad \ssrule{C, s_1}{\skipp, s'}}{\srule{\awaitprotectend{B}{C}, s}{\skipp, s'}}{}
	\end{align*}
	\begin{align*}
		&\infer{(B-Let-CBN)}{\bbrule{\subs{E}{P}{x}}{n}}{\bbrule{\letin{x = P}{E}}{n}}{} 
		&\infer{(B-Let-CBV)}{\bbrule{P}{m}\quad \bbrule{\subs{E}{m}{x}}{n}}{\bbrule{\letin{x = P}{E}}{n}}{}\qquad
		&\infer{(B-Fn)}{-}{\srule{\text{fn } x\colon T\ \Rightarrow\ e, s}{\text{fn } x\colon T\ \Rightarrow\ e, s}}{}
	\end{align*}
	\begin{align*}
		&\infer{(B-App-CBN)}{\bbrule{E_1}{\text{fn } x\colon T \Rightarrow E}\qquad \bbrule{\subs{E}{E_2}{x}}{n}}{\bbrule{E_1 E_2}{n}}{}
		&\infer{(B-App-CBV)}{\bbrule{E_1}{\text{fn } x\colon T \Rightarrow E}\qquad \bbrule{E_2}{v}\qquad \bbrule{\subs{E}{v}{x}}{n}}{\bbrule{E_1 E_2}{n}}{}
	\end{align*}

	\newpage
	
\section*{Semantica small-step}
\subsection*{Grammatica delle espressioni}
\begin{minipage}{.45\linewidth}
	\begin{flushleft}
	\begin{align*}
		op\ :\colon =&\ +\ \big|\ \geq \\ \\
		e\in \textup{Exp}\ :\colon =&\ n\ \big|\ b\ \big|\ e\ \textup{op}\ e\ \big|\ \textup{if}\ e\ \textup{then}\ e\ \textup{else}\ e\ \big|\ \textup{skip}\ \big|\ e;e\ \big|\ \textup{while}\ e\ \textup{do}\ e\ \big|\ e_1 \colon = e_2\ \big|\ !e\ \big|\ \text{ref } e\ \big|\ l\ \big|\
		 \text{fn } x\colon T\ \Rightarrow\ e\ \big|\ e_1 e_2\ \big|\  \text{let } x\colon T = e\ \textup{in } e\ \big|\ e\oplus e\ \big|\ e\| e\ \big|\\ 
		 &\textup{await}\ e\ \textup{protect}\ e\ \textup{end}\ \big| \ \# \text{lab } e
	\end{align*}
	\end{flushleft}
\end{minipage}
\subsection*{Regole per la semantica}
%% ADJUST WIDTH
	\begin{align*}
		&\infer{(S-Left)}{E_1 \goesto E_1' }{E_1 + E_2 \goesto E_1' + E_2}{} 
		&\infer{(S-N.Right)}{E_2 \goesto E_2'}{n_1 + E_2 \goesto n_1 + E_2'}{} \qquad\qquad
		&\infer{(S-Add)}{-}{n_1+n_2 \goesto_{ch} n_3}{n_3 = add(n_1, n_2)} \\ \\
		&\infer{(S-Left)}{E_1 \goesto_{ch} E_1'}{E_1 + E_2 \goesto_{ch} E_1' + E_2}{} 
		&\infer{(S-Right)}{E_2 \goesto_{ch} E_2'}{E_1 + E_2 \goesto_{ch} E_1 + E_2'}{} \qquad\qquad
		&\infer{(op+)}{-}{\srule{n_1 + n_2,s}{n,s}}{n=add(n_1,n_2)} \\ \\
		&\infer{(op-geq)}{-}{\srule{n_1 \geq n_2,s}{b,s}}{b=geq(n_1, n_2)} 
		&\infer{(op1)}{\srule{e_1,s}{e_1', s'}}{\srule{e_1 + e_2,s}{e_1' + e_2,s'}}{} \qquad\qquad
		&\infer{(op2)}{\srule{e_2,s}{e_2', s'}}{\srule{v + e_2,s}{v + e_2',s'}}{} \\ \\
		&\infer{(op1b)}{\srule{e_2,s}{e_2', s'}}{\srule{e_1 + e_2,s}{e_1 + e_2',s'}}{} 
		&\infer{(op2b)}{\srule{e_1,s}{e_1', s'}}{\srule{e_1 + v,s}{e_1' + v',s'}}{} \qquad\qquad
		&\infer{(deref1)}{-}{\srule{!l,s}{v,s}}{\text{if } l \in dom(s) \wedge s(l)=v} \\ \\
		&\infer{(deref2)}{\srule{e,s}{e',s'}}{\srule{!e,s}{!e',s'}}{} 
		&\infer{(ref1)}{-}{\srule{\text{ref }v, s}{l, \memrep{s}{l}{v}}}{l \notin dom(s)} \qquad\qquad
		&\infer{(ref2)}{\srule{e,s}{e',s'}}{\srule{\text{ref }e,s}{\text{ref } e',s}}{}\\ \\
		&\infer{(assign1)}{-}{\srule{l:=v,s}{\text{skip}, \memrep{s}{l}{v}}}{\text{if } l \in dom(s)} 
		&\infer{(assign2)}{\srule{e,s}{e',s'}}{\srule{l:=e,s}{l:=e',s'}}{} \qquad\qquad
		&\infer{(assign3)}{\srule{e_1,s}{e_1',s'}}{\srule{e_1:=e_2,s}{e_1':=e_2,s'}}{}
	\end{align*}
	\begin{align*}
		&\infer{(if-tt)}{-}{\srule{\text{if } true \text{ then } e_1 \text{ else } e_2,s}{e_1,s}}{}
		&\infer{(if-ff)}{-}{\srule{\text{if } false \text{ then } e_1 \text{ else } e_2,s}{e_2,s}}{}\qquad
		&\infer{(if)}{\srule{e,s}{e',s'}}{\srule{\text{if } e \text{ then } e_1 \text{ else } e_2,s}{\text{if } e' \text{ then } e_1 \text{ else } e_2,s'}}{}
	\end{align*}
	\begin{align*}
		\infer{(while)}{-}{\srule{\text{while } e \text{ do } e_1, s}{\text{if } e \text{ then } (e_1; \text{while } e \text{ do } e_1) \text{ else } skip, s}}{}\qquad
		\infer{(assign1b)}{-}{\srule{l:=n,s}{n, \memrep{s}{l}{n}}}{l \in dom(s)} \\ \\
		\infer{(seq.skip)}{-}{\srule{skip;e_2,s}{e_2,s}}{}\qquad
		\infer{(seq)}{\srule{e_1,s}{e_1',s}}{\srule{e_1;e_2,s}{e_1';e_2,s'}}{}\qquad\qquad
		\infer{(seq.skipb)}{-}{\srule{v;e_2,s}{e_2,s}}{}
	\end{align*}
	\newpage
	\begin{minipage}{.45\linewidth}
		\begin{flushleft}
			\begin{align*}
				&\infer{(record1)}{\srule{e_i, s}{e_i', s'}}{\srule{\{\text{lab}_1 = v_1, \dots, \text{lab}_i = e_i, \dots, \text{lab}_k = e_k \}, s}{\{\text{lab}_1 = v_1, \dots, \text{lab}_i = e_i', \dots, \text{lab}_k = e_k \}, s'}}{}
			\end{align*}
			\begin{align*}
				&\infer{(record2)}{-}{\srule{\# \text{lab}_i\ \{\text{lab}_1 = v_1, \dots, \text{lab}_i = e_i, \dots, \text{lab}_k = e_k \}, s}{v_i, s}}{}
				&\infer{(record3)}{\srule{e, s}{e', s'}}{\srule{\# \text{lab } e, s}{\# \text{lab } e', s'}}{}
			\end{align*}
			\begin{align*}
				&\infer{(par-L)}{\srule{e_1,s}{e_1', s'}}{\srule{e_1\| e_2, s}{e_1'\| e_2 , s'}}{}
				&\infer{(par-R)}{\srule{e_2,s}{e_2', s'}}{\srule{e_1\| e_2, s}{e_1\| e_2' , s'}}{}\qquad\qquad
				&\infer{(end-L)}{-}{\srule{\textup{skip}\| e, s}{e , s}}{}
				&\infer{(end-R)}{-}{\srule{e \| \textup{skip}, s}{e , s}}{} \\ \\
				&\infer{(await)}{\ssrule{e_1, s}{\textup{true}, s'}\qquad \ssrule{e_2, s'}{\textup{skip}, s''}}{
					\srule{\textup{await}\ e_1\ \textup{protect}\ e_2\ \textup{end}, s}{\textup{skip}, s''}}{}  
				&\infer{(ChoiceL)}{\srule{e_1, s}{e_1', s'}}{\srule{e_1 \oplus e_2, s}{e_1', s'}}{} \qquad\qquad
				&\infer{(ChoiceR)}{\srule{e_2, s}{e_2', s'}}{\srule{e_1 \oplus e_2, s}{e_2', s'}}{} 
			\end{align*}
		
\subsection*{Call-By-Value}
	\begin{align*}
		&\infer{(CBV-app1)}{\srule{e_1, s}{e_1', s'}}{\srule{e_1 e_2, s}{e_1'e_2, s'}}{} \qquad
		\infer{(CBV- app2)}{\srule{e_2, s}{e_2', s'}}{\srule{ve_2, s}{ve_2', s'}}{} \qquad
		\infer{(CBV-fn)}{-}{\srule{(fn\ x\colon T\Rightarrow e)v, s}{\subs{e}{v}{x}, s}}{} \\ \\
		&\infer{(CBV-let1)}{\srule{e_1,s}{e_1', s'}}{\srule{\letin{x\colon T = e_1}{e_2, s}}{\letin{x\colon T = e_1'}{e_2, s'}}}{} \qquad
		\infer{(CBV-let2)}{-}{\srule{\letin{x\colon T = v}{e_2, s}}{\subs{e_2}{v}{x}, s}}{} \\ \\
		&\infer{(CBV-fix)}{e\equiv\ \text{fn } f\colon T_1 \rightarrow T_2\ \Rightarrow\ e_2}{\srulea{\text{fix}.e}{e(\text{fn }x\colon T_1 \Rightarrow\ (\text{fix}.e)\ x)}}{}
	\end{align*}
\subsection*{Call-By-Name}
	\begin{align*}
		&\infer{(CBN-app)}{\srule{e_1, s}{e_1', s'}}{\srule{e_1e_2, s}{e_1'e_2, s'}}{} \qquad
		\infer{(CBN-fn)}{-}{\srule{(fn\ x\colon T\Rightarrow e)e_2, s}{\subs{e}{e_2}{x},s}}{} \\ \\
		&\infer{(CBN-let)}{-}{\srule{\letin{x\colon T = e_1}{e_2, s}}{\subs{e_2}{e_1}{x}, s}}{} \\ \\
		&\infer{(CBN-fix)}{-}{\srulea{\text{fix}.e}{e(\text{fix}.e)}}{}
	\end{align*}
	\end{flushleft}
\end{minipage}
	\newpage
	
\section*{Grammatica dei tipi e type system}
\vspace{-.5cm}
\begin{minipage}{.45\linewidth}
	\begin{flushleft}
	\begin{align*}
		T&:\colon=\ \integer\ \big|\ \bool\ \big|\ \textup{unit}\ \big|\ T_1\rightarrow T_2\ \big|\ T_1 + T_2\ \big|\ T_1 * T_2\ 
		\big|\ \textup{ref}\ T\ \big|\ \{lab_1\colon T_1, \dots, lab_k\colon T_k \}\ \big|\ \text{proc} \\
		T_{\text{loc}}&:\colon = \text{ref } T
	\end{align*}	
	\vspace{-1cm}
\subsection*{Tipi primitivi e operatori}
	\begin{align*}
		&\infer{(int)}{-}{\Gamma \vdash n\colon \integer}{\ \textup{for}\ n \in \Z} \qquad
		\infer{(bool)}{-}{\Gamma \vdash b\colon \bool}{\ \textup{for}\ b \in \{true, false\}} \qquad
		\infer{(not)}{\Gamma \vdash E\colon \bool}{\Gamma \vdash \neg E\colon \bool} \qquad
		\infer{(op +)}{\Gamma \vdash e_1\colon \integer\qquad \Gamma \vdash e_2\colon \integer}{\Gamma\vdash e_1 + e_2 \colon \integer}{}\qquad
		\infer{(op *)}{\Gamma \vdash e_1\colon \integer\qquad \Gamma \vdash e_2\colon \integer}{\Gamma\vdash e_1 * e_2 \colon \integer}{} \\ \\
		&\infer{(op or)}{\Gamma\vdash e_1\colon \bool\qquad \Gamma\vdash e_2\colon\bool}{\Gamma\vdash e_1\ or\ e_2 \colon\bool}{}\qquad\qquad
		\infer{(op and)}{\Gamma\vdash e_1\colon \bool\qquad \Gamma\vdash e_2\colon\bool}{\Gamma\vdash e_1\ and\ e_2 \colon\bool}{} \qquad\qquad
		\infer{(op $ \geq $)}{\Gamma \vdash e_1\colon \integer\qquad \Gamma \vdash e_2\colon \integer}{\Gamma\vdash e_1 \geq e_2 \colon \bool}{}\qquad\qquad
		%&\infer{(assign)}{\Gamma \vdash e\colon \integer}{\Gamma\vdash l := e\colon \textup{unit}}{\ \textup{if}\ \Gamma(l) =\textup{intref}} 
		\infer{(skip)}{-}{\Gamma\vdash skip\colon unit}{}\\ \\
		&\infer{(seq)}{\Gamma \vdash e_1\colon unit \qquad \Gamma\vdash e_2\colon T}{\Gamma \vdash e_1;e_2\colon T}{}\qquad\qquad
		\infer{(if)}{\Gamma\colon e_1\colon\bool\quad\Gamma\vdash e_2\colon T\quad \Gamma\vdash e_3 \colon T}{\Gamma\vdash \textup{if}\ e_1\ \textup{then}\ e_2\ \textup{else}\ e_3\colon T }{} \qquad\qquad
		\infer{(while)}{\Gamma\vdash e_1 \colon \bool\qquad \Gamma\vdash e_2 \colon T}{\Gamma \vdash\textup{while}\ e_1\ \textup{do}\ e_2\colon T}{}\qquad\qquad
		\infer{(let)}{\Gamma \vdash e_1\colon T\qquad \Gamma,x\colon T \vdash e_2\colon T'}{
			\Gamma\vdash \textup{let}\ x\colon T = e_1\ \textup{in}\ e_2\colon T'}{} \\ \\
		&\infer{(T-fix)}{\Gamma\vdash e\colon (T_1\rightarrow T_2)\rightarrow(T_1\rightarrow T_2)}{\Gamma\vdash \text{fix}.e\colon T_1\rightarrow T_2}{}
	\end{align*}
\subsection*{Referenze}
	\begin{align*}
		&\infer{(ref)}{\Gamma\vdash e \colon T}{\Gamma\vdash \text{ref } e\colon \text{ref } T}{}  \qquad  \qquad
		\infer{(deref)}{\Gamma\vdash e \colon\text{ref } T}{\Gamma\vdash !e \colon T}{} \qquad
		\infer{(assign)}{\Gamma\vdash e_1\colon \text{ref } T\quad \Gamma\vdash e_2\colon T }{\Gamma\vdash (e_1 \colon = e_2)\colon \text{unit}}{}\qquad
		\infer{(loc)}{-}{\Gamma\vdash l\colon \text{ref } T}{\ \Gamma (l) = \text{ref } T} \qquad
	\end{align*}
	\begin{align*}
	\infer{(derefInt)}{-}{\Gamma\vdash !l \colon \integer}{\Gamma (l) = \text intref} \qquad
	\infer{(assignInt)}{\Gamma\vdash e\colon \integer }{\Gamma\vdash (l \colon = e)\colon \text{unit}}{\Gamma (l) = \text intref}\qquad
	\infer{(locInt)}{-}{\Gamma\vdash l\colon \text intref}{\ \Gamma (l) = \text intref} \qquad
	\end{align*}
\subsection*{Funzioni}
	\begin{align*}
		&\infer{(var)}{-}{\Gamma \vdash x \colon T}{\ \textup{if}\ \Gamma (x) = T} 
		&\infer{(fn)}{\Gamma , x \colon T \vdash e\colon T'}{\Gamma\vdash (fn\ x\colon T\ \Rightarrow\ e )\colon T\rightarrow T'}{} \qquad\qquad
		&\infer{(app)}{\Gamma \vdash e_1 \colon T \rightarrow T'\qquad \Gamma\vdash e_2\colon T}{\Gamma\vdash e_1 e_2\colon T'}{} 
	\end{align*}
\subsection*{Record}
	\begin{align*}
		&\infer{(record)}{\Gamma \vdash e_1 \colon T_1\ \dots\ \Gamma \vdash e_k \colon T_k}{\Gamma \vdash \{\text{lab}_1 = e_1, \dots, \text{lab}_k = e_k \}\ \colon \{\text{lab}_1 \colon T_1,\dots, \text{lab}_k\colon T_k  \} }{}
		&\infer{(recordproj)}{\Gamma\vdash e\colon \{\text{lab}_1\colon T_1,\dots, \text{lab}_k\colon T_k \} }{\Gamma\vdash \# \text{lab}_i\ e \colon T_i}{}
	\end{align*}

		
	\end{flushleft}
\end{minipage}
\begin{minipage}{.45\linewidth}
	\begin{flushleft} 
    \subsection*{Concorrenza}
	\begin{align*}
		&\infer{(T-sq1)}{\Gamma\vdash e_1\colon\textup{unit}\qquad \Gamma\vdash e_2\colon\textup{unit}}{\Gamma\vdash e_1;e_2\colon \textup{unit}}{} 
		&\infer{(T-sq2)}{\Gamma\vdash e_1\colon\textup{proc}\qquad \Gamma\vdash e_2\colon\textup{proc}}{\Gamma\vdash e_1;e_2\colon \textup{proc}}{} \qquad\qquad
		&\infer{(T-par)}{\Gamma\vdash e_1\colon T_1\qquad \Gamma\vdash e_2\colon T_2}{\Gamma\vdash e_1 \| e_2\colon \textup{proc}}{\ T_1, T_2 \in \{\textup{unit}, \textup{proc}\}} \\ \\
		&\infer{(T-await)}{\Gamma\vdash e_1 \colon \bool\qquad \Gamma\vdash e_2\colon \textup{unit}}{
		\Gamma\vdash \textup{await}\ e_1\ \textup{protect}\ e_2\ \textup{end}\colon \textup{unit} }{} 
		&\infer{(T-choice)}{\Gamma\vdash e_1 \colon \textup{unit}\qquad \Gamma\vdash e_2\colon \textup{unit}}{\Gamma\vdash e_1 \oplus e_2\colon \textup{unit}}{}\qquad
		%\end{align*}
		%\begin{align*}
		&\infer{(T-lock)}{-}{\Gamma\vdash lock\ m\colon \textup{unit}}\qquad
		\infer{(T-unlock)}{-}{\Gamma\vdash unlock\ m\colon \textup{unit}}\qquad
		\end{align*}
		\section*{Sottotipaggio}
	\begin{align*}
		&\infer{(sub)}{\Gamma\vdash e \colon T\qquad T<\colon T'}{\Gamma\vdash e\colon T'}{}
		&\infer{(s-refl)}{-}{T<\colon T}{} \qquad\qquad
		&\infer{(s-trans)}{T<\colon T'\qquad T'<\colon T''}{T<\colon T''}{}
	\end{align*}
\subsection*{Sottotipaggio dei record}
	\begin{align*}
		&\infer{(rec-perm)}{\pi\ \textup{una permutazione di } 1,2,\dots, k}{
			\{p_1\colon T_1, \dots,p_k\colon T_k\}<\colon \{p_{\pi(1)}\colon T_{\pi (1)},\dots, p_{\pi(k)}\colon T_{\pi(k)} \}}{}
		&\infer{(rec-width)}{-}{\{p_1\colon T_1,\dots,p_k\colon T_k, p_{k+1}\colon T_{k+1},\dots, p_z\colon T_z \} <\colon \{p_1\colon T_1,\dots, p_k\colon T_k \}}{} \\ \\
		&\infer{(rec-depth)}{T_1<\colon T_1'\ \dots\ T_k<\colon T_k'}{\{ p_1\colon T_1,\dots , p_k\colon T_k \} <\colon \{ p_1\colon T_1',\dots,p_k\colon T_k'\}}{}
	\end{align*}
	
\subsection*{Sottotipaggio delle funzioni}
			\[
				\infer{(fun-sub)}{T_1\colon> T_1'\qquad T_2<\colon T_2'}{T_1\rightarrow T_2 <\colon T_1'\rightarrow T_2'}{}
			\]
\subsection*{Sottotipaggio somma e prodotto}
	\begin{align*}
		&\infer{(prod-sub)}{T_1<\colon T_1'\qquad T_2<\colon T_2'}{T_1 \ast T_2 <\colon T_1'\ast T_2'}{} 
		&\infer{(sum-sub)}{T_1<\colon T_1'\qquad T_2<\colon T_2'}{T_1 + T_2 <\colon T_1'+ T_2'}{}
	\end{align*}
	\end{flushleft}
\end{minipage}	

\newpage


\section*{Progress}


\begin{thm}[Progress]
	Sia $ e $ chiusa, $ \Gamma \vdash e\colon T$ e $ dom(\Gamma) \subseteq dom(s) $ allora o $ e $ è un valore oppure $ \exists e',s'$ t.c. $ \srule{e, s}{e', s'} $.
\end{thm}
\begin{prf}
	Sia $ \phi $ una relazione ternaria definita come segue:
$
		\phi(\Gamma, e, T)\ \stackrel{def}{=} \ \forall s. \text{dom}(\Gamma) \subseteq \text{dom}(s)\ \Rightarrow\ \text{value}(e)\ \vee\ (\exists e', s'\ t.c.\ \srule{e, s}{e', s'})
$

 	Proviamo che per ogni $ \Gamma, e, T,\ $ se $\ \Gamma \vdash e\colon T\ $ allora $\ \phi (\Gamma, e, T) $ usando \textit{rule induction} su $\ \Gamma \vdash e\colon T $.
\end{prf}
\begin{multicols}{2}
Provare la progress sul seguente linguaggio:

\[
	E:\colon = x \ \big|\ n \ \big|\ E_1 + E_2 \ \big|\ E_1;E_2; \ \big|\ \textup{fn}\ x\colon T \Rightarrow E \ \big|\ EE\ 
	\big|\ \textup{while}\ E\ \textup{do}\ E\ \big|\ E_1\colon = E_2 \ \big|
\]


\begin{prf}[by \textit{rule induction} su come ho tipato $ E\ (\Gamma \vdash E\colon T) $]
	\begin{base}[int]
		\[
			\infer{(int)}{-}{\Gamma \vdash n\colon \integer}{\ \textup{for}\ n \in \Z}
		\]
		Supponiamo di aver derivato $ \Gamma \vdash E\colon T $ utilizzando la regola di tipaggio \textbf{(int)}. Posso concludere che $ E\equiv n $, per $ n \in \Z $, e $ T\equiv \text{int} $. Ne consegue che $ \phi (\Gamma, E, T) = \text{true} $ perché $ \text{value}(n) = \text{true} $.
	\end{base}
	\begin{base}[bool]
		\[
		\infer{(bool)}{-}{\Gamma \vdash b\colon \bool}{\ \textup{for}\ b \in \{true, false\}}
		\]
		Supponiamo di aver derivato $ \Gamma \vdash E\colon T $ utilizzando la regola di tipaggio \textbf{(bool)}.\\ 
		Posso concludere che $ E\equiv b $, per $ b \in \{\text{true}, \text{false}\} $, e $ T\equiv \text{bool} $.\\ 
		Ne consegue che $ \phi (\Gamma, E, T) = \text{true} $ perché $ \text{value}(b) = \text{true} $.
	\end{base}
	\begin{base}[skip]
		\[
			\infer{(skip)}{-}{\Gamma\vdash \skipp\colon unit}{}
		\]
		Supponiamo di aver derivato $ \Gamma \vdash E\colon T $ utilizzando la regola di tipaggio \textbf{(skip)}. Posso concludere che $ E\equiv \skipp $, $ T\equiv unit $. Ne consegue che $ \phi (\Gamma, \skipp, unit) = \textup{true} $ perchè $ \text{value}(\skipp) = \text{true} $.
	\end{base}
	\begin{base}[var]
		\[
			\infer{(var)}{-}{\Gamma \vdash x \colon T}{\ \textup{if}\ \Gamma (x) = T} 
		\]
		Supponiamo di aver derivato $ \Gamma \vdash E\colon T $ utilizzando la regola di tipaggio \textbf{(var)}. Posso concludere che $ E\equiv x $, e $ T\equiv T_1\rightarrow T_2$ oppure $ T\equiv \text{int} $. Posso dire che $ E $ non è chiusa perché $ x \in FV(E) $. Dato che la premessa della dimostrazione è falsa posso dedurre che $ \phi (\Gamma, E, T) = \text{true} $.
	\end{base}
	\begin{ind}[fn]
		\[
			\infer{(fn)}{\Gamma , x \colon T \vdash e\colon T'}{\Gamma\vdash (fn\ x\colon T\ \Rightarrow\ e )\colon T\rightarrow T'}{}
		\]
		Supponiamo di aver derivato $ \Gamma \vdash E\colon T $ utilizzando la regola di tipaggio \textbf{(fn)}. Posso concludere che $ E\equiv fn\ x\colon T\ \Rightarrow\ e $, $ T\equiv T\rightarrow T' $ e $ \Gamma , x \colon T \vdash e\colon T' $.\\ Ne consegue che $ \phi (\Gamma, E, T) = \text{true} $ perché $ \text{value}(fn\ x\colon T\ \Rightarrow\ e) = \text{true} $.
	\end{ind}
	\begin{ind}[op +]
		\[
			\infer{(op +)}{\Gamma \vdash e_1\colon \integer\qquad \Gamma \vdash e_2\colon \integer}{\Gamma\vdash e_1 + e_2 \colon \integer}{}
		\]
		Supponiamo di aver derivato $ \Gamma \vdash E\colon T $ utilizzando la regola di tipaggio \textbf{(op +)}. Posso concludere che $ E\equiv e_1 + e_2 $, $ T\equiv \text{int} $, $ \Gamma\vdash e_1 \colon \text{int} $ e $ \Gamma\vdash e_2 \colon \text{int} $. Per \textbf{ipotesi induttiva} ho che $ \phi(\Gamma, e_1, int)= \text{true} $ e $ \phi(\Gamma, e_2, int)= \text{true} $. Perciò contraddistinguo tre casi:
		\begin{enumerate}
			\item 
			\[
				\infer{(op1)}{\srule{e_1,s}{e_1', s'}}{\srule{e_1 + e_2,s}{e_1' + e_2,s'}}{}
			\]
			Se $ \text{value}(e_1) = \text{false} $ allora $ \langle e_1, s \rangle $ fa un passo e quindi, applicando la regola della semantica small-step \textbf{(op1)}, abbiamo che $ \phi(\Gamma, e_1 + e_2, \text{int}) = \textup{true} $;
			\item
			\[
				\infer{(op+)}{-}{\srule{n_1 + n_2,s}{n,s}}{n=add(n_1,n_2)}
			\]
			Se $ \text{value}(e_1) = \text{true}\ \wedge\ \text{value}(e_2) = \text{true} $ allora, applicando la regola della semantica small-step \textbf{(op +)}, abbiamo che $ \phi(\Gamma, e_1 + e_2, \text{int}) = \textup{true} $;
			\item 
			\[
				\infer{(op2)}{\srule{e_2,s}{e_2', s'}}{\srule{v + e_2,s}{v + e_2',s}}{}
			\]
			Se $ \text{value}(e_1) = \text{true}\ \wedge\ \text{value}(e_2) = \text{false} $ allora, applicando la regola della semantica small-step \textbf{(op2)}, abbiamo che $ \phi(\Gamma, e_1 + e_2, \text{int}) = \textup{true} $.
		\end{enumerate}
	\end{ind}
	\begin{ind}[seq]
		\[
			\infer{(seq)}{\Gamma \vdash e_1\colon unit \qquad \Gamma\vdash e_2\colon T}{\Gamma \vdash e_1;e_2\colon T}{}
		\]
		Supponiamo di aver derivato $ \Gamma \vdash E\colon T $ utilizzando la regola di tipaggio \textbf{(seq)}. Posso concludere che $ E\equiv e_1;e_2 $, $ T\equiv \text{unit} $, $ \Gamma\vdash e_1 \colon \text{unit} $ e $ \Gamma\vdash e_2 \colon \text{unit} $.  Per \textbf{ipotesi induttiva} ho che $ \phi(\Gamma, e_1, unit)= \text{true} $ e $ \phi(\Gamma, e_2, unit)= \text{true} $. Perciò contraddistinguo due casi:
		\begin{enumerate}
			\item
			\[
				\infer{(seq)}{\srule{e_1,s}{e_1',s}}{\srule{e_1;e_2,s}{e_1';e_2,s'}}{}
			\]
			Se $ \text{value}(e_1) = \text{false} $ allora $ \langle e_1, s \rangle $ fa un passo e quindi, applicando la regola della semantica small-step \textbf{(seq)}, abbiamo che $ \phi(\Gamma, e_1;e_2, \text{unit}) = \textup{true} $;
			\item
			\[
				\infer{(seq.skip)}{-}{\srule{skip;e_2,s}{e_2,s}}{}
			\]
			Se $ \text{value}(e_1) = \text{true} $ allora, applicando la regola della semantica small-step \textbf{(seq.skip)}, abbiamo che $ \phi(\Gamma, e_1;e_2, \text{unit}) = \textup{true} $.
		\end{enumerate}
	\end{ind}
	\begin{ind}[while]
		\[
			\infer{(while)}{\Gamma\vdash e_1 \colon \bool\qquad \Gamma\vdash e_2 \colon T}{\Gamma \vdash\textup{while}\ e_1\ \textup{do}\ e_2\colon T}{}
		\]
		Supponiamo di aver derivato $ \Gamma \vdash E\colon T $ utilizzando la regola di tipaggio \textbf{(while)}. Posso concludere che $ E\equiv \textup{while}\ e_1\ \textup{do}\ e_2 $, per qualche $ e_1, e_2 $ tali che $ \Gamma\vdash e_1 \colon \bool $ e $ \Gamma\vdash e_2 \colon T $, e $ T\equiv unit $. Ne consegue che $ \phi(\Gamma, E, T) = \textup{true} $ perché applico la regola di riscrittura, applicando la regola semantica \textbf{(while)} e quindi faccio un passo:
		$ \ \langle e, s \rangle \equiv \srule{\textup{while}\ e_1\ \textup{do}\ e_2}{\text{if } e \text{ then } (e_1; \text{while } e \text{ do } e_1) \text{ else } skip} \equiv \langle e', s'\rangle $.
	\end{ind}
	\columnbreak
	\begin{ind}[assign]
		\[
			\infer{(t-assign)}{\Gamma\vdash e_1\colon \text{int}}{\Gamma\vdash l\colon = e_1\colon unit}{if \Gamma(l) = intref}
		\]
		Supponiamo di aver derivato $ \Gamma\vdash E\colon T $ utilizzando la regola di tipaggio \textbf{(T-assign)}. Posso concludere che $ E\equiv l\colon = e_1 $ per qualche locazione $ l $ e qualche programma $ e_1 $ tali che $ \Gamma\vdash e_1 \colon \text{int} $, e $ T\equiv \text{unit} $ e $ \Gamma(l) = intref $. Ne consegue che, per il fatto che l'albero di derivazione di $ \Gamma\vdash e_1\colon \text{int} $ ha profondità minore di $ \Gamma\vdash E\colon T $, per ipotesi induttiva $ \phi(\Gamma, e_1, unit) =\text{true} $. Contraddistinguo quindi due casi:
		\begin{enumerate}
			\item
			\[
				\infer{(assign1)}{-}{\srule{l:=v,s}{\text{skip}, \memrep{s}{l}{v}}}{\text{if } l \in dom(s)} 
			\]
			Se $ \text{value}(e_1) = \text{true} $ abbiamo che $ e_1 = n $ per qualche $ n $, $ E\equiv l\colon = n $ per qualche $ l $. Dal fatto che $ \{l\} \subseteq \text{dom}(\Gamma)\subseteq \text{dom}(s) $, derivo che $ s $ è definito in $ l $. Posso quindi applicare la regola semantica \textbf{(assign1)} e quindi ho che $ \phi(\Gamma, E, T) =\text{true} $ perché faccio un passo: 
			$ \langle e, s \rangle \equiv \srule{l\colon = n, s}{\skipp, s[l\mapsto n]}\equiv \langle e', s' \rangle $.
			\item
			\[
				\infer{(assign2)}{\srule{e_1,s}{e_1',s'}}{\srule{l:=e_1,s}{l:=e_1',s}}{}
			\]
			Se $ \text{value}(e_1) = \text{false} $ abbiamo che $ E\equiv l\colon = e_1 $ per qualche $ l $. Quindi $ \forall s.\ \text{dom}(\Gamma)\subseteq \text{dom}(s)\ \srule{e_1, s}{e_1', s'} $ per qualche $ e_1' $. Posso quindi applicare la regola semantica \textbf{(assign2)} e quindi ho che $ \phi(\Gamma, E, T) =\text{true} $ perché faccio un passo:
			$ \langle e, s \rangle \equiv \srule{l\colon= e_1, s}{l\colon= e_1', s'} \equiv \langle e', s' \rangle$
		\end{enumerate} 
	\end{ind}
\end{prf}
	
\end{multicols}

\newpage
\setcounter{thm}{0}
\setcounter{prf}{0}
\setcounter{base}{0}
\setcounter{ind}{0}
\section*{Type Preservation}
\begin{thm}[Type Preservation]
	Sia $ e $ chiusa e $\ \Gamma \vdash e\colon T\ $ e $\ \text{dom}(\Gamma) \subseteq \text{dom}(s)\ $ e $\ \srule{e, s}{e', s'} \ $ allora $\ \Gamma \vdash e'\colon T\ $ e $\ e' $ è chiusa e $\ \text{dom}(\Gamma) \subseteq \text{dom}(s') $.
\end{thm}

\begin{multicols}{2}
Provare la type preservation sul seguente linguaggio:

\[
E:\colon = x \ \big|\ n \ \big|\ E_1 + E_2 \ \big|\ \textup{fn}\ x\colon T \Rightarrow E \ \big|\ EE
\]


	
\begin{prf}[by \textit{rule induction} su come ho derivato $ E\rightarrow F$]
	\begin{base}[S-Add]
		\[
			\infer{(S-Add)}{-}{\srule{n_1 + n_2,s}{n_3,s}}{n_3=add(n_1,n_2)}
		\]
		Supponiamo di aver derivato $ E\rightarrow F $ utilizzando la regola semantica \textbf{(S-Add)}. Ne consegue che $ E\equiv n_1 + n_2 $ e $ F\equiv n_3 $. Se $ \Gamma\vdash E\colon \text{int} $ sarò stato in grado di tipare $ E $ utilizzando la regola di tipaggio \textbf{(op +)}. Di conseguenza ho tipato $ n_1 $ e $ n_2 $: $ \ \Gamma\vdash n_1 \colon \text{int} $, 
		$ \ \Gamma\vdash n_2 \colon \text{int} $. Ne consegue che $ \ \Gamma\vdash n_3\colon \text{int} $ come richiesto.
	\end{base}
	\begin{base}[CBV-fn]
		\[
			\infer{(CBV-fn)}{-}{\srule{(fn\ x\colon T\Rightarrow e)v, s}{\subs{e}{v}{x}, s}}{}
		\]
		Supponiamo di aver derivato $ E\rightarrow F $ utilizzando la regola semantica \textbf{(CBV-fn)}. Ne consegue che $ E\equiv (fn\ x\colon T\Rightarrow e)v $ e $ F\equiv \subs{e}{v}{x} $. Se $ \Gamma\vdash E\colon T $ sarò stato in grado di tipare $ E $ utilizzando la regola di tipaggio \textbf{(app)}. In tal caso avrò dedotto che $ \Gamma \vdash v\colon T'$ e $ \Gamma\vdash fn\ x\colon T'\Rightarrow e \colon T\rightarrow T' $. Di conseguenza sarò stato in grado di tipare il corpo di $ e $: $ \ \Gamma, x\colon T'\vdash e\colon T $. Dunque per il \textit{substitution lemma} avrò che $ \Gamma\vdash\subs{e}{v}{x}\equiv F\colon T $ come richiesto.
	\end{base}
	\begin{ind}[S-Left]
		\[
			\infer{(S-Left)}{E_1 \goesto E_1' }{E_1 + E_2 \goesto E_1' + E_2}{} 
		\]
		Supponiamo di aver derivato $ E\rightarrow F $ utilizzando la regola semantica \textbf{(S-Left)}. Ne consegue che $ E\equiv E_1 + E_2 $ e $ F\equiv E_1' + E_2 $. Se $ \Gamma\vdash E\colon T $ allora vuol dire che sarò stato in grado di tipare $ E $ usando la regola di tipaggio \textbf{(op +)}. In tal caso avrò dedotto che $ T\equiv \text{int} $, $ \Gamma\vdash E_1\colon \text{int} $, $ \Gamma\vdash E_2\colon \text{int} $. Per ipotesi induttiva da $ E_1\rightarrow E_1' $ concludo che $ \Gamma\vdash E_1'\colon \text{int} $. Ma allora applicando la regola di tipaggio \textbf{(op +)} derivo che $ \Gamma\vdash E_1' + E_2\equiv F \colon \text{int} $ come richiesto.
	\end{ind}
	\columnbreak
	\begin{ind}[S-Right]
		\[
			\infer{(S-Right)}{E_2 \goesto E_2'}{E_1 + E_2 \goesto E_1 + E_2'}{}
		\]
		Supponiamo di aver derivato $ E\rightarrow F $ utilizzando la regola semantica \textbf{(S-Right)}. Ne consegue che $ E\equiv E_1 + E_2 $ e $ F\equiv E_1 + E_2' $. Se $ \Gamma\vdash E\colon T $ allora vuol dire che sarò stato in grado di tipare $ E $ usando la regola di tipaggio \textbf{(op +)}. In tal caso avrò dedotto che $ T\equiv \text{int} $, $ \Gamma\vdash E_1\colon \text{int} $, $ \Gamma\vdash E_2\colon \text{int} $. Per ipotesi induttiva da $ E_2\rightarrow E_2' $ concludo che $ \Gamma\vdash E_2'\colon \text{int} $. Ma allora applicando la regola di tipaggio \textbf{(op +)} derivo che $ \Gamma\vdash E_1 + E_2'\equiv F \colon \text{int} $ come richiesto.
	\end{ind}
	\begin{ind}[CBV-app1]
		\[
			\infer{(CBV-app1)}{\srule{e_1, s}{e_1', s'}}{\srule{e_1 e_2, s}{e_1'e_2, s'}}{}
		\]
		Supponiamo di aver derivato $ E\rightarrow F $ utilizzando la regola semantica \textbf{(CBV-app1)}. Ne consegue che $ E\equiv E_1E_2 $ e $ F\equiv E_1'E_2 $ e $ E_1\rightarrow E_1' $. Se $ \Gamma\vdash E\colon T $ allora vuol dire che sarò stato in grado di tipare $ E $ usando la regola di tipaggio \textbf{(app)}. Quindi per ipotesi induttiva da $ E_1\rightarrow E_1' $ deduco che $ \Gamma\vdash E_1'\colon T_A\rightarrow T $. Siccome $ F\equiv E_1'E_2 $ e $ \Gamma \vdash E_2 \colon T_A $, ne consegue che, applicando la regola di tipaggio \textbf{(app)}, derivo che $ \Gamma\vdash E_1'E_2\equiv F \colon T $ come richiesto.
	\end{ind}
	\begin{ind}[CBV-app2]
		\[
			\infer{(CBV- app2)}{\srule{e_2, s}{e_2', s'}}{\srule{ve_2, s}{ve_2', s'}}{}
		\]
		Supponiamo di aver derivato $ E\rightarrow F $ utilizzando la regola semantica \textbf{(CBV-app2)}. Ne consegue che $ E\equiv vE_2 $ e $ F\equiv vE_2' $ e $ E_2\rightarrow E_2' $. Se $ \Gamma\vdash E\colon T $ allora vuol dire che sarò stato in grado di tipare $ E $ usando la regola di tipaggio \textbf{(app)}. Quindi per ipotesi induttiva da $ E_2\rightarrow E_2' $ deduco che $ \Gamma\vdash E_2'\colon T_A $. Siccome $ F\equiv vE_2' $ e $ \Gamma \vdash v \colon T_A\rightarrow T $, ne consegue che, applicando la regola di tipaggio \textbf{(app)}, derivo che $ \Gamma\vdash vE_2'\equiv F \colon T $ come richiesto.
	\end{ind}
\end{prf}
\end{multicols}



\end{document}