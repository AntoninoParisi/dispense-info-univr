\documentclass[a4paper, 10pt]{article}
\usepackage[italian]{babel}
\usepackage[T1]{fontenc}
\usepackage[utf8]{inputenc}
\usepackage{amsmath}
\usepackage{amsfonts}
\usepackage{amsthm}
\usepackage{graphicx}
\usepackage[sans]{frontespizio}
\usepackage{hyperref}
\hypersetup{hidelinks,
	colorlinks = true,
	urlcolor = black, 
	linkcolor = black}
\usepackage[T1]{fontenc}
\usepackage[utf8]{inputenc}
\usepackage[margin=3cm]{geometry}
\usepackage{multicol}
\usepackage{booktabs}
\usepackage{fancyhdr}
\usepackage{tikz}
\usetikzlibrary{calc}
\usepackage{listings}
\lstset{language = SQL,
	basicstyle=\ttfamily,
	showstringspaces=false}
\usepackage{fancyvrb}
\pagestyle{fancy}

\begin{document}
	\begin{frontespizio}
		\Preambolo{\usepackage{datetime}}
		\Istituzione{Università degli Studi di Verona}
		\Divisione{Dipartimento di informatica}
		\Facolta{Scienze e Ingegneria}
		\Scuola{Laurea in Informatica}
		\Titolo{Basi di Dati}
		\Sottotitolo{Programma di laboratorio}
		\Candidato{Davide Bianchi}
		\NCandidato{Autori}
		\Annoaccademico{2016/2017}
	\end{frontespizio}
	
	\tableofcontents
	
	\newpage
	
	\section{Gestione base di dati con Postgresql}
	Di seguito si trova una panoramica dei comandi Postgres più comuni per la gestione di una base di dati.
	
	\subsection{Comando CREATE TABLE}
	Il comando \lstinline{CREATE TABLE} è usato per creare tabelle nella base di dati.
	La sintassi generale è:
	\begin{lstlisting}
CREATE TABLE nomeTabella (
	nomeAttributo dominioAttributo vincoli,
	...
);
	\end{lstlisting}
	dove \lstinline|nomeAttributo| è il nome dell'attributo nella tabella, \lstinline|dominioAttributo| è il dominio dell'attributo da aggiungere alla tabella.
	
	\subsubsection{Domini elementari}
	I domini di default disponibili in Postgres sono:
	\begin{itemize}
		\item \lstinline|BOOLEAN|: valori booleani (true/false);
		\item \lstinline|INTEGER|: valori interi a 4 byte;
		\item \lstinline|SMALLINT|: valori interi a 2 bit;
		\item \lstinline|NUMERIC(p, s)|: valori decimali approssimati, dove \lstinline|p| è la precisione del numero e \lstinline|s| la scala (numero di cifre decimali dopo la virgola);
		\item \lstinline|DECIMAL(p, s)|: valori deciamli approssimati, con i parametri uguali a \lstinline|NUMERIC|.
		\item \lstinline|REAL|: valori in virgola mobile a 6 cifre decimali;
		\item \lstinline|DOUBLE PRECISION|: valori in virgola mobile approssimati a 15 cifre decimali;

	\end{itemize}

	\subsubsection{Domini di caratteri}
	\begin{itemize}
		\item \lstinline|CHARACTER|: singoli caratteri;
		\item \lstinline|CHARACTER(n)|:stringa di caratteri di lunghezza n;
		\item \lstinline|VARCHAR|: stringhe di caratteri di lunghezza variabile;
		\item \lstinline|TEXT|: testo libero (solo Postgres).
	\end{itemize}

	\subsubsection{Domini di bit/booleani}
	\begin{itemize}
		\item \lstinline|BIT|: singoli bit;
		\item \lstinline|VARBIT(n)|: stringa di bit di lunghezza fissa;
		\item \lstinline|VARBIT|: stringa di bit di lunghezza arbitraria.
		\item \lstinline|BOOLEAN|: valori booleani, possono essere solo singoli.
	\end{itemize}
	\textbf{Nota:} non sono ammesse stringhe di booleani.
	
	\subsubsection{Domini di tempo}
	\begin{itemize}
		\item \lstinline|DATE|: date rappresentate tra apici e nel formato \verb|YYYY-MM-DD|;
		\item \lstinline|TIME(precisione)|: misure di tempo nel formato \verb|hh:mm:ss:[precisione]|;
		\item \lstinline|INTERVAL|: intervalli di tempo.
		\item \lstinline|TIME/TIMESTAMP WITH TIME ZONE|: con tutti i dati per il tempo, con indicazioni sul fuso.
	\end{itemize}
	
	\subsection{Comando CREATE DOMAIN}
	Questo comando è usato per creare un dominio utente \textbf{invariabile nel tempo}.
	\begin{lstlisting}
CREATE DOMAIN nome AS tipoBase [default]
	[vincolo]
	\end{lstlisting}
	I valori di default e i vincoli sono opzionali.
	
\end{document}