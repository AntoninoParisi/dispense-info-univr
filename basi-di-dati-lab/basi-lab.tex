\documentclass[a4paper, 10pt]{article}
\usepackage[italian]{babel}
\usepackage[T1]{fontenc}
\usepackage[utf8]{inputenc}
\usepackage{amsmath}
\usepackage{amsfonts}
\usepackage{amsthm}
\usepackage{graphicx}
\usepackage[dvipsnames]{xcolor}
\usepackage[sans]{frontespizio}
\usepackage{hyperref}
\hypersetup{hidelinks,
	colorlinks = true,
	urlcolor = black, 
	linkcolor = black}
\usepackage[T1]{fontenc}
\usepackage[utf8]{inputenc}
\usepackage[margin=3cm]{geometry}
\usepackage{booktabs}
\usepackage{fancyhdr}
\usepackage{tikz}
\usetikzlibrary{calc}
\usepackage{listings}
\lstset{language = SQL,
	basicstyle=\ttfamily,
	showstringspaces=false,
	keywordstyle=\color{blue}\bfseries,
	morekeywords={REFERENCES, BOOLEAN, REAL, DOUBLE, PRECISION, TEXT, VARBIT},
	stringstyle=\color{Purple}
}
\usepackage{fancyvrb}
\pagestyle{fancy}
\lhead{\nouppercase{\leftmark}}
\rhead{\nouppercase{\rightmark}}

\begin{document}
	\begin{frontespizio}
		\Preambolo{\usepackage{datetime}}
		\Istituzione{Università degli Studi di Verona}
		\Divisione{Dipartimento di informatica}
		\Facolta{Scienze e Ingegneria}
		\Scuola{Laurea in Informatica}
		\Titolo{Basi di Dati}
		\Sottotitolo{Programma di laboratorio}
		\Candidato{Davide Bianchi}
		\Candidato{Matteo Danzi}
		\NCandidato{Autori}
		\Annoaccademico{2016/2017}
	\end{frontespizio}

	\tableofcontents
		
	\newpage
	
	\section{Gestione base di dati con Postgresql}
	Di seguito si trova una panoramica dei comandi Postgres più comuni per la gestione di una base di dati.
	
	\subsection{Comando CREATE TABLE}
	Il comando \lstinline{CREATE TABLE} è usato per creare tabelle nella base di dati.
	La sintassi generale è:
	\begin{lstlisting}
CREATE TABLE nomeTabella (
	nomeAttributo dominioAttributo vincoli,
	...
);
	\end{lstlisting}
	dove \lstinline|nomeAttributo| è il nome dell'attributo nella tabella, \lstinline|dominioAttributo| è il dominio dell'attributo da aggiungere alla tabella.
	
	\subsubsection{Domini elementari}
	I domini di default disponibili in Postgres sono:
	\begin{itemize}
		\item \lstinline|BOOLEAN|: valori booleani (true/false);
		\item \lstinline|INTEGER|: valori interi a 4 byte;
		\item \lstinline|SMALLINT|: valori interi a 2 bit;
		\item \lstinline|NUMERIC(p, s)|: valori decimali approssimati, dove \lstinline|p| è la precisione del numero (cifre a sinistra e a destra della virgola) e \lstinline|s| la scala (numero di cifre decimali dopo la virgola);
		\item \lstinline|DECIMAL(p, s)|: valori deciamli approssimati, con i parametri uguali a \lstinline|NUMERIC|.
		\item \lstinline|REAL|: valori in virgola mobile a 6 cifre decimali;
		\item \lstinline|DOUBLE PRECISION|: valori in virgola mobile approssimati a 15 cifre decimali;
	\end{itemize}
	\textbf{Nota:} Se si devono rappresentare importi di denaro che contengono anche
	decimali, \textbf{MAI} usare \lstinline|REAL| o \lstinline|DOUBLE PRECISION| ma usare \lstinline|NUMERIC|!

	\subsubsection{Domini di caratteri}
	\begin{itemize}
		\item \lstinline|CHARACTER|: singoli caratteri;
		\item \lstinline|CHARACTER(n)|:stringa di caratteri di lunghezza n;
		\item \lstinline|VARCHAR|: stringhe di caratteri di lunghezza variabile;
		\item \lstinline|TEXT|: testo libero (solo Postgres).
	\end{itemize}

	\subsubsection{Domini di bit/booleani}
	\begin{itemize}
		\item \lstinline|BIT|: singoli bit;
		\item \lstinline|VARBIT(n)|: stringa di bit di lunghezza fissa;
		\item \lstinline|VARBIT|: stringa di bit di lunghezza arbitraria.
		\item \lstinline|BOOLEAN|: valori booleani, possono essere solo singoli.
	\end{itemize}
	\textbf{Nota:} non sono ammesse stringhe di booleani.
	
	\subsubsection{Domini di tempo}
	\begin{itemize}
		\item \lstinline|DATE|: date rappresentate tra apici e nel formato \verb|YYYY-MM-DD|;
		\item \lstinline|TIME(precisione)|: misure di tempo nel formato \verb|hh:mm:ss:[precisione]|;
		\item \lstinline|INTERVAL|: intervalli di tempo.
		\item \lstinline|TIME/TIMESTAMP WITH TIME ZONE|: con tutti i dati per il tempo, con indicazioni sul fuso.
	\end{itemize}
	
	\subsection{Comando CREATE DOMAIN}
	Questo comando è usato per creare un dominio utente \textbf{invariabile nel tempo}.
	\begin{lstlisting}
CREATE DOMAIN nome AS tipoBase [default]
	[vincolo]
	\end{lstlisting}
	I valori di default e i vincoli sono opzionali.
	
	\textbf{Esempio:}
	\begin{lstlisting}
CREATE DOMAIN giorniSettimana AS CHAR(3)
	CHECK ( VALUE IN ('LUN', 'MAR', 'MER', 'GIO', 'VEN', 
	'SAB', 'DOM'));
	\end{lstlisting}

	\subsection{Vincoli di attributo e di tabella}
	Vincoli di attributo/intrarelazionali specificano proprietà che devono
	essere soddisfatte da ogni tupla di una singola relazione della base di
	dati.
	
	\begin{lstlisting}
[ CONSTRAINT vincolo ]
{ NOT NULL |
  CHECK ( espressione ) [ NO INHERIT ] |
  DEFAULT valore |
  UNIQUE |
  PRIMARY KEY |
  REFERENCES tabella [ ( attributo ) ]
    [ ON DELETE azione ] [ ON UPDATE azione ] }
	\end{lstlisting}
	
	Vincoli di tabella:
	\begin{lstlisting}
[ CONSTRAINT vincolo ]
{ CHECK ( espressione ) [ NO INHERIT ] |
  UNIQUE ( attributo [, ... ]) |
  PRIMARY KEY ( attributo [, ... ]) |
  FOREIGN KEY ( attributo [, ... ])
  REFERENCES reftable [ ( refcolumn [ , ... ]) ]
	[ ON DELETE azione ] [ ON UPDATE azione ] }
	\end{lstlisting}

	\begin{itemize}
		\item \lstinline|NOT NULL|: determina che il valore nullo non è ammesso come
		valore dell’attributo.
		\item \lstinline|DEFAULT valore|: specifica un valore di
		default per un attributo quando un comando di inserimento dati non
		specifica nessun valore per l’attributo.
		
	\textbf{Esempio:}
	\begin{lstlisting}
nome VARCHAR (20) NOT NULL,
cognome VARCHAR (20) NOT NULL DEFAULT ''
	\end{lstlisting}
		\item \lstinline|UNIQUE|: impone che i valori di un attributo (o di un insieme di
		attributi) siano una \textbf{superchiave}.
		\item \lstinline|PRIMARY KEY|: identifica l’attributo che rappresenta la chiave
		primaria della relazione:
		\begin{itemize}
			\item Si usa una sola volta per tabella.
			\item Implica il vincolo \lstinline|NOT NULL|.
		\end{itemize}
	
	\textbf{Esempio: }
	\begin{lstlisting}
matricola CHAR(6) PRIMARY KEY;
	\end{lstlisting}
	oppure su più attributi
	\begin{lstlisting}
nome VARCHAR(20),
cognome VARCHAR(20),
PRIMARY KEY(nome, cognome)
	\end{lstlisting}
	
	\item \lstinline|CHECK (vincolo)|: specifica un vincolo generico che devono soddisfare le
	tuple della tabella.
	\end{itemize}
	
	\subsubsection{Vincoli di integrità referenziale}
		Un vincolo di integrità referenziale si dichiara nella tabella interna e ha
		due possibili sintassi.
		\begin{itemize}
			\item \lstinline|REFERENCES|: \textbf{vincolo di attributo}, da usare quando il vincolo è su un
			singolo attributo della tabella interna, $ |A| = 1 $.
			\item \lstinline|FOREIGN KEY|: \textbf{vincolo di tabella} , da usare quando il vincolo coinvolge più attributi della tabella interna, $ |A| > 1 $.
		\end{itemize}
		\textbf{Esempio:}
		\begin{lstlisting}
CREATE TABLE Interna
...
attributo VARCHAR(15) REFERENCES TabellaEsterna (chiave)
...

...
piano VARCHAR (10),
stanza INTEGER,
FOREIGN KEY (piano, stanza) REFERENCES Ufficio (piano, nStanza)
		\end{lstlisting}
		
	\subsection{Comando ALTER TABLE}
		La struttura di una tabella si può modificare dopo la sua creazione con il
		comando \lstinline|ALTER TABLE|.
		\begin{itemize}
			\item Aggiunta di un nuovo attributo con \lstinline|ADD COLUMN|:
			\begin{lstlisting}
ALTER TABLE impiegato ADD COLUMN stipendio NUMERIC(8,2);
			\end{lstlisting}
			\item Rimozione di un attributo con \lstinline|DROP COLUMN|:
			\begin{lstlisting}
ALTER TABLE impiegato DROP COLUMN stipendio;
			\end{lstlisting}
			\item Modifica di un valore di default di un attributo con \lstinline|ALTER COLUMN|:
			\begin{lstlisting}
ALTER TABLE impiegato ALTER COLUMN stipendio
	SET DEFAULT 1000.00;
			\end{lstlisting}			
		\end{itemize}
		
	\subsection{Comando INSERT INTO}
		Una tabella viene popolata con il comando \lstinline|INSERT INTO|:
		\begin{lstlisting}
INSERT INTO impiegato (matricola, nome, cognome)
	VALUES ('A00001', 'Mario', 'Rossi'),
	       ('A00002', 'Luca', 'Bianchi');
		\end{lstlisting}
	
	\subsection{Comando UPDATE}
		Una tupla di una tabella può essere modificata con il comando \lstinline|UPDATE|:
		\begin{lstlisting}
UPDATE tabella
	SET attributo = espressione [, ... ]
	[ WHERE condizione ];
		\end{lstlisting}
	\lstinline|condizione| è una espressione booleana che seleziona quali righe
	aggiornare. Se \lstinline|WHERE| non è presente, tutte le tuple saranno aggiornate.
	
	\textbf{Esempio: }
	\begin{lstlisting}
UPDATE impiegato
	SET stipendio = stipendio * 1.10
	WHERE nomeDipartimento = 'Vendite';
UPDATE impiegato
	SET telefono = '+39' || telefono;
	\end{lstlisting}
	\textbf{Nota:} L’operatore '||' concatena due espressioni e ritorna la stringa corrisp.
	
	\subsection{Comando DELETE}
		Le tuple di una tabella vengono cancellate con il comando \lstinline|DELETE|:
		\begin{lstlisting}
DELETE FROM impiegato WHERE matricola = 'A001';
		\end{lstlisting}
		Una tabella viene cancellata con il comando \lstinline|DROP TABLE|.
		
	\subsection{Politiche di reazione}
		In SQL si possono attivare diverse politiche di adeguamento della tabella
		interna
		\begin{lstlisting}
FOREIGN KEY ( column_name [ , ... ]) REFERENCES
	reftable [ ( refcolumn [ , ... ]) ]
   ON DELETE reazione ON UPDATE reazione
		\end{lstlisting}
		\begin{itemize}
			\item \lstinline|CASCADE|: la modifica del valore di un attributo riferito nella tabella master
			si propaga anche in tutte le righe corrispondenti nelle tabelle slave.

			\item \lstinline|SET NULL|: la modifica del valore di un attributo riferito nella tabella
			master determina che in tutte le righe corrispondenti nelle tabelle slave
			il valore dell’attributo referente è posto a \lstinline|NULL| (se ammesso).

			\item \lstinline|SET DEFAULT|: la modifica del valore di un attributo riferito nella tabella
			master determina che in tutte le righe corrispondenti nelle tabelle slave
			il valore dell’attributo referente è posto al valore di default (se esiste).

			\item \lstinline|NO ACTION|: indica che non si fa nessuna azione. Il vincolo però deve
			essere sempre valido. Quindi, la modifica del valore di un attributo
			riferito nella tabella master non viene effettuata.
		\end{itemize}
	\subsection{Query sul database}
\end{document}